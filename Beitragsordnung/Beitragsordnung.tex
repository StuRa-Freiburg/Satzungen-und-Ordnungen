\documentclass[fontsize=12pt,parskip=half]{scrartcl}
\usepackage[ngerman]{babel}
\usepackage[T1]{fontenc}
\usepackage{lmodern}
\usepackage{textcomp}
\usepackage{enumerate}


\usepackage[clausemark=forceboth, juratotoc, juratocnumberwidth=2.5em]{scrjura}
\useshorthands{'}
\defineshorthand{'S}{\Sentence\ignorespaces}
\defineshorthand{'.}{. \Sentence\ignorespaces}

\usepackage{hyperref}

\hypersetup{
  colorlinks=false,
  hidelinks
}

\pagestyle{myheadings}

\begin{document}
\subject{Lesefassung}
\title{Beitragsordnung}
\subtitle{der Studierendenschaft der Albert-Ludwigs-Universität Freiburg}
\date{Stand: 04.12.2013}
\maketitle

\pagebreak

\tableofcontents

\vspace*{\fill}

Auf Grund von § 65a Abs. 1 Satz 1, Abs. 3 Satz 2 und Abs. 5 Sätze 2 bis 5 des
Landeshochschulgesetzes (LHG) vom 1. Januar 2005 (GBl. S. 1), zuletzt geändert
durch Artikel 2 Verfasste-Studierendenschafts-Gesetz vom 10. Juli 2012 (GBl. S.
457), sowie der §§ 3 Abs. 4 und 10 Abs. 2 der Organisationssatzung der
Studierendenschaft der Universität Freiburg (Organisationssatzung) vom 17. Mai
2013 (Amtliche Bekanntmachungen der Albert-Ludwigs-Universität Freiburg Jhg. 44
Nr. 28 vom 17.05.2013) hat der Studierendenrat der Studierendenschaft der
Universität Freiburg am 26.11.2013 die nachstehende Beitragsordnung der
Studierendenschaft der Universität Freiburg beschlossen.

In dieser Ordnung wird grundsätzlich das Gendersternchen (*) verwendet. Dieses
soll die Vielfalt der Ausprägungen besonders menschlicher Sexualität in all
ihren Dimensionen versinnbildlichen und stellt eine deutliche Positionierung
gegen die Reproduktion patriarchaler Strukturen vor allem über eine sprachliche
Indifferenz im Zuge einer rhetorischen Modernisierung der
Geschlechterverhältnisse dar.

\pagebreak

\begin{contract}
  \Clause{title={Beitragszweck}}
  'S Die Verfasste Studierendenschaft der Universität Freiburg
  (Studierendenschaft) nimmt als eine rechtsfähige Körperschaft des öffentlichen
  Rechts und Gliedkörperschaft der Universität Freiburg unbeschadet der
  Zuständigkeit der Universität Freiburg und des Studierendenwerks
  Freiburg-Schwarzwald Aufgaben nach § 65 Abs. 2 LHG wahr. 

  'S Um diese Aufgaben erfüllen zu können, erhebt die Studierendenschaft gemäß §
  65a Abs. 5 Sätze 2 bis 5 LHG von den Studierenden Beiträge nach Maßgabe dieser
  Beitragsordnung.

  \Clause{title={ Beitragspflicht }}
  'S Die Studierendenschaft erhebt zur Erfüllung ihrer Aufgaben von allen
  immatrikulierten Studierenden (§ 60 Abs. 1 Satz 1 LHG) und immatrikulierten
  Doktoranden und Doktorandinnen (§ 38 Abs. 5 Satz 2 LHG) der Universität
  Freiburg (Studierende) einen Studierendenschaftsbeitrag'. Der Beitragspflicht
  unterliegen auch die vom Studium beurlaubten Studierenden, nicht jedoch die
  befristet eingeschriebenen ausländischen Studierenden im Sinne des § 60 Abs. 1
  S. 2 LHG.

  'S Der Beitrag ist pro Semester zu zahlen.

  \Clause{title={ Beitragshöhe }}
  'S Der von den Studierenden ab dem Sommersemester 2014 zu zahlende
  Studierendenschaftsbeitrag beträgt 7,00 Euro für jedes Semester.

  \Clause{title={ Befreiung, Erlass, Ermäßigung, Stundung und Erstattung des Beitrags }}
  'S Befreiungen vom Studierendenschaftsbeitrag sind nicht vorgesehen'. Der
  Studierendenschaftsbeitrag kann nicht erlassen, nicht ermäßigt und nicht
  gestundet werden'. Bei der Festsetzung der Beitragshöhe wurden die sozialen
  Belange der Studierenden berücksichtigt.

  'S Bei Exmatrikulation binnen vier Wochen nach Beginn der Vorlesungszeit des
  Semesters für das der Beitrag erhoben wird, wird der
  Studierendenschaftsbeitrag erstattet'. Der Vollzug der Erstattung erfolgt
  durch die Universität'. Diese behält bis zum Ablauf der in Satz 1 genannten
  Frist einen entsprechenden Anteil der Studierendenschaftsbeiträge, der für
  Rückerstattungen nach vorzeitiger Exmatrikulation vorgesehen ist, ein'. Die
  Restmittel werden nach Ablauf der in Satz 1 genannten Frist an die
  Studierendenschaft abgeführt.

  \Clause{title={Fälligkeit des Beitrags, Einzug und Rechtsfolgen nicht
    fristgerechter Zahlung des Beitrags}}
  'S Der Studierendenschaftsbeitrag wird erstmals mit der Immatrikulation oder
  Rückmeldung zum Sommersemester 2014 fällig'. In diesem Zusammenhang gelten die
  von der Universität Freiburg gesetzten Fristen.

  'S Der Studierendenschaftsbeitrag für das bevorstehende Semester wird jeweils
  mit Beginn der von der Universität Freiburg für die Immatrikulation oder
  Rückmeldung festgesetzten Frist fällig, ohne dass es des Erlasses eines
  Beitragsbescheides bedarf, und ist innerhalb dieser Frist gemäß § 65a Abs. 5
  Satz 5 LHG an die Universität Freiburg zu zahlen'. Die Universität führt einen
  Teil dieser Beiträge jeweils zum Ende der Rückmeldefrist (20. März und 20.
  September) an die Verfasste Studierendenschaft ab'. Den Restbetrag behält die
  Universität Freiburg bis zum in § 4 Abs. 2 Satz 1 genannten Zeitpunkt ein.
  Dieser Anteil der Studierendenschaftsbeiträge ist für Rückerstattungen bei
  Exmatrikulation binnen vier Wochen nach Vorlesungsbeginn vorgesehen'. Die
  Restmittel werden vier Wochen nach Beginn der Vorlesungszeit des Semesters für
  das der Beitrag erhoben wird an die Studierendenschaft abgeführt.

  'S Wird der Studierendenschaftsbeitrag nicht rechtzeitig gezahlt, erhebt die
  Universität Freiburg auf Grundlage dieser Ordnung und nach Maßgabe der
  Gebührensatzung der Albert-Ludwigs-Universität (Amtliche Bekanntmachungen, Jg.
  38 Nr. 3, S. 8 f. vom 23. Januar 2007) eine Säumnisgebühr.

  'S Die Universität Freiburg hat einer Person die Immatrikulation gemäß § 60
  Abs. 5 Nr. 2 LHG zu versagen, soweit diese den fälligen
  Studierendenschaftsbeitrag nicht innerhalb der für die Immatrikulation
  festgesetzten Frist an die Universität Freiburg entrichtet hat.

  'S Studierende sind von der Universität Freiburg gemäß § 62 Abs. 2 Nr. 3 LHG
  von Amts wegen zu exmatrikulieren, wenn sie den Studierendenschaftsbeitrag
  trotz Mahnung und Androhung der Exmatrikulation nach Ablauf der für die
  Zahlung gesetzten Frist nicht gezahlt haben.

  \Clause{title={ Änderung der Beitragsordnung }}
  'S Die Beitragsordnung kann mit der absoluten Mehrheit der Mitgliederstimmen
  des Studierendenrates geändert werden.

  \Clause{title={ Inkrafttreten }}
  'S Die Beitragsordnung tritt am Tage nach ihrer Veröffentlichung in den
  Amtlichen Bekanntmachungen der Albert-Ludwigs-Universität Freiburg in Kraft. 
\end{contract}
\end{document}