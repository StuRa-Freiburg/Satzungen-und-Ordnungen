\documentclass[fontsize=12pt,parskip=half]{scrartcl}
\usepackage[ngerman]{babel}
\usepackage{lmodern}
\usepackage{textcomp}
\usepackage{enumerate}

\usepackage{iftex}
\ifLuaTeX
\else
\usepackage[T1]{fontenc}
\fi

\usepackage[clausemark=forceboth, juratotoc, juratocnumberwidth=2.5em]{scrjura}
\useshorthands{'}
\defineshorthand{'S}{\Sentence\ignorespaces}
\defineshorthand{'.}{. \Sentence\ignorespaces}

\usepackage{hyperref}

\hypersetup{
  colorlinks=false,
  hidelinks
}

\pagestyle{myheadings}

\begin{document}
\subject{Lesefassung}
\title{Organisationssatzung}
\subtitle{der Studierendenschaft der Albert-Ludwigs-Universität Freiburg}
\date{Stand: 24.10.2018}
\maketitle

\pagebreak

\tableofcontents

\vspace*{\fill}

Aufgrund des Verfasste-Studierendenschafts-Gesetzes vom 13.07.2012 gibt sich die
Studierendenschaft der Albert-Ludwigs-Universität Freiburg in Urabstimmung vom
29.04., 30.04. und 02.05.2013, geändert durch die erste Satzung zur Änderung der
Organisationssatzung der Verfassten Studierendenschaft vom 02.07.2014, die
zweite Satzung zur Änderung der Organisationssatzung vom 26.02.2015 und die
dritte Satzung zur Änderung der Organisationssatzung vom 26.06.2018, folgende
Organisationssatzung.\\
Das Rektorat der Albert-Ludwigs-Universität Freiburg hat die letzte Änderung
dieser Organisationssatzung am 24.10.2018 genehmigt.

In dieser Ordnung wird grundsätzlich das Gendersternchen (*) verwendet. Dieses
soll die Vielfalt der Ausprägungen besonders menschlicher Sexualität in all
ihren Dimensionen versinnbildlichen und stellt eine deutliche Positionierung
gegen die Reproduktion patriarchaler Strukturen vor allem über eine sprachliche
Indifferenz im Zuge einer rhetorischen Modernisierung der
Geschlechterverhältnisse dar.

\pagebreak

\addsec{Präambel}

Von 1977 bis 2012 waren die Studierendenvertretungen durch die CDU-geführten
Regierungen des Landes Baden-Württemberg mundtot gemacht und gegängelt worden.
Unzählige Engagierte versuchten in den unabhängigen Studierendenvertretungen,
trotz dieser widrigen Bedingungen den Anliegen der Studierenden Gehör in
Hochschule und Gesellschaft zu verschaffen. Im Bewusstsein der damaligen
Zustände sind die Studierenden der Albert-Ludwigs-Universität Freiburg
aufgefordert, für ihre Belange einzutreten, an der politischen Willensbildung
mitzuwirken und sich für die Durchsetzung der Demokratie einzusetzen. Zentrales
Mittel dafür ist die Studierendenschaft der Albert-Ludwigs-Universität Freiburg
samt ihrer Organe, die ausschließlich den Interessen der Studierenden
verpflichtet ist.
Die Studierendenschaft der Albert-Ludwigs-Universität Freiburg setzt sich
entsprechend ihrer gesellschaftlichen Verpflichtung für die Belange der
Studierenden, die freie Entfaltung des Individuums, Gleichstellung,
interkulturelle Verständigung, die Pflege der Beziehung zu
Studierendenorganisationen im In- und Ausland sowie die Anwendung von
Forschungsergebnissen ausschließlich zu friedlichen Zwecken ein. Sie wendet sich
gegen Diskriminierung.


\addsec{Abschnitt I: Studierendenschaft}

\begin{contract}
  \Clause{title={Die Studierendenschaft}}
  \label{Abs1:Stud}

  'S Die Studierendenschaft der Albert-Ludwigs-Universität Freiburg
  (Studierendenschaft) ist eine rechtsfähige Körperschaft des öffentlichen
  Rechts'. Sie ist Gliedkörperschaft der Albert-Ludwigs-Universität Freiburg'. Sie
  gliedert sich in Fachschaften, die sich in Fachbereiche gliedern'. Sie hat
  Organe auf Fachbereichsebene und zentraler Ebene. \label{Abs1:Stud.Beschreibung}

  'S Die Studierendenschaft vertritt die Studierenden der
  Albert-Ludwigs-Universität Freiburg'. Sie verwaltet ihre Angelegenheiten im
  Rahmen der gesetzlichen Bestimmungen selbst'. Sie hat gemäß § 65 Absatz 2 LHG
  unbeschadet der Zuständigkeit der Hochschule und des Studentenwerks die
  folgenden Aufgaben: \label{Abs1:Stud:Vertretung}
  \begin{enumerate}[\qquad 1.]
    \item die Wahrnehmung der hochschulpolitischen, fachlichen und
      fachübergreifenden sowie der sozialen, wirtschaftlichen und kulturellen
      Belange der Studierenden,
    \item die Mitwirkung an den Aufgaben der Hochschulen nach den §§ 2 bis 7
      des Landeshochschulgesetzes,
    \item die Förderung der politischen Bildung und des staatsbürgerlichen
      Verantwortungsbewusstseins der Studierenden,
    \item die Förderung der Gleichstellung und den Abbau von Benachteiligungen
      innerhalb der Studierendenschaft, insbesondere hinsichtlich Geschlecht,
      sexueller Identität, sexueller Orientierung, Behinderung, chronischer
      Krankheit, sozialer Herkunft, ethnischer Zugehörigkeit, Weltanschauung,
      familiärer Verpflichtungen und altersspezifischer Bedürfnisse,
    \item die Förderung der sportlichen Aktivitäten der Studierenden,
    \item die Pflege der regionalen, überregionalen und internationalen
      Studierendenbeziehungen und
    \item ie Herstellung des Einvernehmens bei der Verwendung der
      Qualitätssicherungsmittel gemäß § 3 Qualitätssicherungsgesetz, nach
      Maßgabe der Grundordnung der Universität.
  \end{enumerate}
  \label{Abs1:Stud.polMan}
  'S Im Rahmen der Erfüllung ihrer Aufgaben nimmt die Studierendenschaft ein
  politisches Mandat wahr'. Sie wahrt nach den verfassungsrechtlichen Grundsätzen
  die weltanschauliche, religiöse und parteipolitische Neutralität.

  'S Zur Erfüllung ihrer Aufgaben ermöglicht die Studierendenschaft den
  Meinungsaustausch in der Gruppe der Studierenden und kann insbesondere auch zu
  solchen Fragen Stellung beziehen, die sich mit der gesellschaftlichen
  Aufgabenstellung der Hochschule, ihrem Beitrag zur nachhaltigen Entwicklung
  sowie mit der Anwendung der wissenschaftlichen Erkenntnisse und der
  Abschätzung ihrer Folgen für die Gesellschaft und die Umwelt beschäftigen'. Sie
  kann hierzu Medien aller Art nutzen und in diesen Medien auch die Diskussion
  und Veröffentlichung zu allgemeinen gesellschaftlichen Fragen im Rahmen ihres
  Mandats ermöglichen. 


  \Clause{title={Organe der Studierendenschaft }}

  'S Die Studierendenschaft beschließt und handelt durch ihre Organe'. Die Organe
  der Studierendenschaft sind
  \begin{enumerate}[\qquad 1.]
    \item die Vollversammlung aller Studierenden (VV),
    \item der Studierendenrat (StuRa) als legislatives Organ,
    \item die Fachbereichsvertretungen (FaVe),
    \item der Allgemeine Studierendenausschuss (AStA) als exekutives Organ,
    \item die Wahl-, Schlichtungs- und Satzungskommission (WSSK).
  \end{enumerate}
  'S Daneben können Sachbeschlüsse auch durch Urabstimmungen gefasst werden.

  'S Über die Ergebnisse der Sitzungen der Organe sind Niederschriften
  anzufertigen, die archiviert und grundsätzlich veröffentlicht werden'. Das
  Nähere regeln die Geschäftsordnungen der jeweiligen Organe'. Von jeder Sitzung,
  auch der Fachbereiche und der Referate, muss als Grundlage für Zahlungen und
  transparente Arbeit ein Ergebnisprotokoll veröffentlicht werden'. Daneben kann
  es auch ein Verlaufsprotokoll geben.

  'S Die Organe der Studierendenschaft haben das Recht, im Rahmen ihrer Aufgaben
  Anträge an die zuständigen Kollegialorgane der Hochschule zu stellen; diese
  sind verpflichtet, sich mit den Anträgen zu befassen.

  'S Die Amtszeit der Mitglieder der Organe dauert vom 1. Oktober bis zum 30.
  September des darauf folgenden Jahres, soweit diese Satzung nichts anderes
  bestimmt'. Sie endet vorzeitig durch Verlust der Mitgliedschaft der
  Studierendenschaft, Abwahl oder Rücktritt'. Bei vorzeitigem Ende der Amtszeit
  verkürzt sich die Amtszeit des*der Nachfolgers*in entsprechend'. Die Wahl- und
  Urabstimmungsordnung hat Stellvertretungs-, Nachrückverfahren und Neuwahl zu
  regeln'. Ist das Innehaben mehrerer Ämter nach dieser Satzung unzulässig, so
  ist das bisherige Amt, vor der Annahme der Wahl in ein weiteres Amt, durch
  Erklärung gegenüber dem Studierendenrat und der WSSK, niederzulegen'. Die
  Mitglieder der Organe führen ihre Geschäfte bis zum Amtsantritt eines*r
  Nachfolgers*in interimsweise fort, es sei denn, sie wurden abgewählt.

  'S Die Mitglieder der Organe üben ihre Tätigkeiten ehrenamtlich aus; § 26 Absatz
  5 bleibt unberührt'. Sie dürfen wegen ihrer Tätigkeit in der Studierendenschaft
  nicht benachteiligt werden'. Die Tätigkeit als Mitglied in den Organen der
  Studierendenschaft während mindestens eines Jahres kann bis zu einem
  Studienjahr bei der Berechnung der Prüfungsfristen unberücksichtigt bleiben;
  die Entscheidung darüber trifft der*die Rektor*in der
  Albert-Ludwigs-Universität Freiburg.


  \Clause{title={Rechte und Pflichten der Mitglieder der Studierendenschaft}}

  'S Mitglieder der Studierendenschaft sind die immatrikulierten Studierenden sowie
  die eingeschriebenen Doktorand*innen der Universität Freiburg'. Diese Satzung
  und die in ihrem Rahmen verabschiedeten Satzungen, Geschäftsordnungen und
  sonstigen Beschlüsse und Maßnahmen sind für alle Mitglieder der
  Studierendenschaft verbindlich.

  'S Im Rahmen dieser Satzung sowie der Wahl- und Urabstimmungsordnung ist jedes
  Mitglied der Studierendenschaft für deren Organe wählbar, wahl- und
  abstimmungsberechtigt.

  'S Jedes Mitglied der Studierendenschaft ist gegenüber allen Organen der
  Studierendenschaft anfrage- und antragsberechtigt'. Es hat grundsätzlich
  Anwesenheits- und Rederecht in den Sitzungen der Organe; Ausnahmen sind in der
  Geschäftsordnung des jeweiligen Organs zu regeln'. Ihm ist auf Verlangen
  unverzüglich Einsicht in die Niederschriften der Sitzungen der Organe zu
  gewähren, soweit ihm nach Satz 2 ein Anwesenheitsrecht zugestanden hätte'.
  Anfragen und Anträge sind unverzüglich an das zuständige Organ weiterzuleiten;
  innerhalb einer in seiner Geschäftsordnung festzulegenden Frist hat es sich
  damit zu beschäftigen und das Ergebnis dem*der Antragstellenden/Anfragenden
  mitzuteilen.

  'S Jedes Mitglied der Studierendenschaft ist verpflichtet, seinen
  Mitgliedsbeitrag rechtzeitig zu entrichten'. Das Nähere regeln die
  Finanzordnung und die Beitragsordnung.

\end{contract}

\addsec{Abschnitt II: Urabstimmung und Vollversammlung}

\begin{contract}

  \Clause{title=Urabstimmung}

  'S Die Urabstimmung ist eine Urnenabstimmung aller Mitglieder der
  Studierendenschaft zu einer Abstimmungsfrage nach § 6 Abs. 2'. In einer
  Urabstimmung können Beschlüsse zu mehreren Abstimmungsfragen gefasst werden.

  'S In einer Urabstimmung kann über alle Angelegenheiten der Studierendenschaft
  ein Beschluss gefasst werden, außer über die Feststellung des
  Haushalts-/Wirtschaftsplans.

  'S Eine Urabstimmung wird durchgeführt, wenn die Abstimmungsfrage von
  \begin{enumerate}[\qquad 1.]
  \item einer Vollversammlung beschlossen wird,
  \item einem Drittel der Stimmen des Studierendenrates beschlossen wird oder
  \item einem Prozent der Mitglieder der Studierendenschaft beantragt wird und
    von der WSSK für zulässig erklärt wurde. Die Urabstimmung muss in der
    Vorlesungszeit stattfinden. Vor der Abstimmung muss eine Vollversammlung
    einberufen werden, auf der die Abstimmungsfrage erörtert wird.
  \end{enumerate}

  'S Für die Durchführung der Urabstimmung ist die WSSK verantwortlich.

  'S Spricht sich die Mehrheit der Abstimmenden für die Abstimmungsfrage aus, ist
  diese beschlossen'. Ein in einer Urabstimmung gefasster Beschluss ist für alle
  Organe der Studierendenschaft verbindlich'. Sofern der Beschluss nicht die
  Anhänge dieser Satzung oder die von dieser Satzung vorgesehenen Satzungen und
  Geschäftsordnungen erlässt, ändert oder aufhebt, kann er innerhalb von zwei
  Jahren nach seiner Bekanntgabe nur durch eine weitere Urabstimmung geändert
  oder aufgehoben werden'. Sofern der Beschluss diese Satzung ändert, kann er
  innerhalb eines Jahres nur durch eine weitere Urabstimmung geändert oder
  aufgehoben werden.

  'S Das Nähere regelt die Wahl- und Urabstimmungsordnung, insbesondere
  \begin{enumerate}[\qquad 1.]
  \item die Mindestdauer der Urnenabstimmung,
  \item die Frist, die zwischen erörternder Vollversammlung und Beginn der
    Urnenabstimmung liegen muss,
  \item bis wann die Abstimmungsfrage und der Zeitraum der Urnenabstimmung
    bekanntgemacht sein müssen.
  \end{enumerate}


  \Clause{title={Vollversammlung aller Studierenden (VV)}}

  'S Die Vollversammlung aller Studierenden ist ein beschließendes Organ'. Alle
  Mitglieder der Studierendenschaft sind rede-, antrags- und stimmberechtigt'.
  Die Vollversammlung kann über alle Angelegenheiten der Studierendenschaft
  beschließen'. Die Vollversammlung kann Beschlüsse zur politischen
  Positionierung der Studierendenschaft fassen.

  'S Die Vollversammlung wird einberufen, wenn dies
  \begin{enumerate}[\qquad 1.]
  \item ein Fünftel der Mitglieder des Studierendenrates beschließt,
  \item der AStA dies mit 2/3-Mehrheit beschließt,
  \item 0,5 Prozent der Mitglieder der Studierendenschaft beantragt oder
  \item zur Debatte über eine Abstimmungsfrage gemäß § 4 Abs. 2 Satz 2 zu
    geschehen hat.
  \end{enumerate}
  'S Der Zeitpunkt der Vollversammlung liegt in der Vorlesungszeit'. Die
  Vollversammlung ist spätestens ein Jahr nach der letzten Vollversammlung
  einzuberufen'. Mindestens zwei Wochen vor ihrer Einberufung müssen die
  Vollversammlung und die Tagesordnungsgegenstände bekanntgemacht werden'. Für
  Bekanntmachung und Einberufung der Vollversammlung ist das
  Studierendenratspräsidium zuständig.

  'S Die Vollversammlung beschließt zu Beginn unabhängig von ihrer
  Beschlussfähigkeit auf Vorschlag des Studierendenratspräsidiums über eine
  Geschäftsordnung, eine Tagesordnung sowie ein Präsidium'. Der
  Tagesordnungsvorschlag muss die nach Absatz 2 beantragten bzw. beschlossenen
  Gegenstände beinhalten'. Änderungen der Tagesordnung auf der Vollversammlung
  sind nur zu nicht bindenden Beschlüssen möglich'. Bis zur Wahl eines Präsidiums
  leitet das Studierendenratspräsidium die Vollversammlung.

  'S Die Beschlussfähigkeit wird zu Beginn festgestellt und muss zum Zeitpunkt
  eines Beschlusses bestehen und in offenkundigen Fällen durch die
  Versammlungsleitung überprüft werden'. Die Vollversammlung ist beschlussfähig,
  wenn zum Zeitpunkt der Feststellung mindestens ein Prozent der Mitglieder der
  Studierendenschaft anwesend sind'. Die WSSK legt die Zahl fest und gibt sie in
  der Studierendenratssitzung vor der Vollversammlung bekannt'. Ist die
  Vollversammlung nicht beschlussfähig, spricht sie Empfehlungen an die anderen
  Organe der Studierendenschaft aus.

  'S Die Vollversammlung beschließt und empfiehlt mit absoluter Mehrheit der
  anwesenden Stimmberechtigten'. Ein Beschluss zu nicht grundsätzlichen
  Angelegenheiten hat Bindungswirkung, sofern ihm kein in Urabstimmung gefasster
  Beschluss entgegensteht'. Ein Beschluss der Vollversammlung zu grundsätzlichen
  Angelegenheiten ist nicht bindend'. Solche Beschlüsse können nur vom
  Studierendenrat gefasst werden.

  'S Empfehlende Beschlüsse und Beschlüsse zu grundsätzlichen Angelegenheiten,
  insbesondere zu Satzungsvorhaben, haben Initiativcharakter'. Die für die
  Beschlüsse der Vollversammlung zuständigen Organe der Studierendenschaft
  müssen diese Beschlüsse spätestens in der zweiten Sitzung nach der
  Vollversammlung verhandeln und entsprechend der jeweiligen Geschäftsordnung
  einen Beschluss dazu fassen.

  'S Ein auf einer Vollversammlung gefasster Beschluss zu nicht grundsätzlichen
  Angelegenheiten kann innerhalb von drei Jahren nach seiner Bekanntgabe nur
  durch eine Urabstimmung oder eine weitere Vollversammlung geändert oder
  aufgehoben werden.

  'S Das Nähere regelt die Geschäftsordnung des Studierendenrates.


  \Clause{title={Direktdemokratische Einflussnahme durch Anträge}}

  'S Die Anträge auf direktdemokratische Einflussnahme nach § 4 Absatz 2 und § 5
  Absatz 2 sowie der Antrag nach § 14 Absatz 3 sind schriftlich unter Angabe
  einer Ansprechperson bei der WSSK einzureichen'. Die Beantragenden müssen
  innerhalb einer Sammelfrist eine Unterschriftenliste der Unterstützer*innen
  der WSSK vorlegen'. Die Sammelfrist beginnt an dem Tag, an dem der Antrag
  gestellt wird'. Außerhalb der Sammelfrist gesammelte Unterschriften sind
  ungültig'. Mehrfache Unterschriften des gleichen Mitglieds der
  Studierendenschaft für den gleichen oder für konkurrierende Anträge machen
  alle Unterschriften dieses Mitglieds ungültig.

  'S Die Beschlüsse und Anträge auf Durchführung einer Urabstimmung müssen eine
  Abstimmungsfrage beinhalten, die nur mit ''Ja'' oder ''Nein"\ beantwortet werden
  kann'. Die WSSK hat die Abstimmungsfrage auf ihre rechtliche Zulässigkeit zu
  prüfen'. Ist die Abstimmungsfrage unzulässig, ist der Beschluss oder Antrag
  nichtig'. Bei Anträgen auf Durchführung einer Urabstimmung verschiebt sich der
  Beginn der Sammelfrist auf den Tag, an dem der Ansprechperson das Ergebnis der
  rechtlichen Prüfung mitgeteilt wird.

  'S Die Beschlüsse und Anträge auf Einberufung einer Vollversammlung müssen
  den/die zu behandelnden Gegenstand/Gegenstände benennen.

  'S Das Nähere regelt die Wahl- und Urabstimmungsordnung, insbesondere
  \begin{enumerate}[\qquad 1.]
  \item die notwendigen Angaben auf der Unterschriftenliste
  \item die Länge der Sammelfrist
  \item die Fristen, innerhalb derer die WSSK das Ergebnis der Prüfung der
    Abstimmungsfrage und der Unterschriftenliste mitteilen muss
  \item   bis wann nach dem Ablauf der Sammelfrist oder nach der
    Beschlussfassung des Organs die Vollversammlung oder Urabstimmung
    stattfinden müssen.
  \end{enumerate}

\end{contract}

\addsec{Abschnitt III: Der Studierendenrat (StuRa)}
\begin{contract}

  \Clause{title={Aufgaben}}

  'S Der Studierendenrat beschließt über alle Angelegenheiten der
  Studierendenschaft, soweit keine bindenden Beschlüsse einer Urabstimmung oder
  Vollversammlung vorliegen'. Er wählt die Mitglieder des AStA und der WSSK; er
  kann die von ihm Gewählten abwählen'. Er spricht Vorschläge für die Besetzung
  der studentischen Sitze in den Gremien der Universität und des Studentenwerks
  aus'. Die vom Studierendenrat gewählten Personen sind verpflichtet sich an die
  Beschlüsse zu halten'. Die gewählten oder vorgeschlagen Personen sind der
  Studierendenschaft Rechenschaft schuldig und sie haben eine Berichtspflicht im
  Studierendenrat'. Soll die Studierendenschaft wirtschaftliche Unternehmen
  gründen oder sich an ihnen beteiligen oder soll sie anderen Organisationen
  beitreten, muss der Studierendenrat zustimmen, bevor sich die
  Studierendenschaft rechtlich bindet'. Die Beschlüsse des Studierendenrates sind
  für die Exekutive verbindlich.

  'S Der Studierendenrat kann die Beschlussfassung über bestimmte Gegenstände auf
  die Exekutive übertragen; davon ausgenommen sind Beschlüsse, die einer
  absoluten oder einer Zwei-Drittel-Mehrheit der Stimmen des Studierendenrates
  bedürfen, die die Gründung von oder die Beteiligung an wirtschaftlichen
  Unternehmen oder den Beitritt zu anderen Organisationen betreffen, sowie
  Wahlen von Mitgliedern des Vorstands und der Referent*innen'. Die Befugnis des
  Studierendenrates, eigene Beschlüsse zum selben Gegenstand zu fällen, wird
  dadurch nicht eingeschränkt.


  \Clause{title={Zusammensetzung}}

  'S Die Mitglieder des Studierendenrats sind die Fachbereichsvertreter*innen nach
  § 17 und zehn weitere Abgeordnete.

  'S Die Abgeordneten werden über eine freie, geheime und gleiche Listenwahl
  gewählt'. Die Anzahl der Abgeordneten, die pro Liste in den Studierendenrat
  gewählt werden, ergibt sich aus dem Adams-Verfahren'. Das Nähere regelt die
  Wahl- und Urabstimmungsordnung.


  \Clause{title={Stimmgewichtung}}

  'S Die Stimmen der Mitglieder des Studierendenrates werden entsprechend der
  Studierendenzahlen ihres jeweiligen Fachbereichs gewichtet.

  'S Fachbereiche mit unter 300 Studierenden haben 2 Stimmen, Fachbereiche mit 300
  bis 1200 Studierenden haben 3 Stimmen und Fachbereiche mit mehr als 1200
  Studierenden haben 4 Stimmen.

  'S Jede*r Abgeordnete*r hat eine Stimme.

  'S Die Stimmen müssen kumuliert abgegeben werden.


  \Clause{title={Beschlussfassung}}

  'S Der Studierendenrat ist beschlussfähig, wenn er ordnungsgemäß einberufen wurde
  und die Mehrheit der Studierendenratsmitglieder anwesend ist'. Die
  Beschlussfähigkeit wird zu Beginn, danach auf Antrag festgestellt'. Der
  Studierendenrat ist beschlussfähig, solange nicht das Gegenteil festgestellt
  wird.

  'S Wird ein Fachbereich in drei Sitzungen in Folge nicht vertreten, so ruht die
  Mitgliedschaft ab dem Ende dieser 3. Sitzung bis der Fachbereich wieder eine*n
  Vertreter*in in den Studierendenrat entsendet'. Ruht die Mitgliedschaft eines
  Fachbereichs, so muss dies durch das Studierendenratspräsidium baldmöglichst
  dem Studierendenrat sowie dem*der Fachbereichsvertreter*in mitgeteilt und in
  der nächsten Studierendenratssitzung bekannt gegeben werden'. So lange die
  Mitgliedschaft ruht, wird der Fachbereich nicht zur Anzahl der zur Berechnung
  der Beschlussfähigkeit und der Mehrheiten herangezogenen Fachbereiche
  hinzugezählt.

  'S Der Studierendenrat beschließt über
  \begin{enumerate}[\qquad 1.]
    \item Änderungen der Organisationssatzung sowie die Wahl und Abwahl von
      WSSK- Mitgliedern mit der Zustimmung der Stimmen von zwei Dritteln ihrer
      Mitglieder (Zwei-Drittel-Mehrheit),
    \item die Wahl der Vorsitzenden und der anderen AStA Mitglieder, die Abwahl
      der von ihr gewählten Personen sowie Erlass, Änderungen und Aufhebung der
      Geschäftsordnung des Studierendenrates, der Zuordnung der Studienfächer zu
      den Fachbereichen nach § 13 Abs. 2 sowie der Finanzordnung und der
      sonstigen Satzungen, insbesondere des Haushalts-/ Wirtschaftsplans, der
      Beitragsordnung, sowie der Wahl- und Abstimmungsordnung mit der Mehrheit
      der Stimmen ihrer Mitglieder (absolute Mehrheit) und 3. alle anderen
      Angelegenheiten mit der Mehrheit der abgegebenen Stimmen ohne
      Berücksichtigung der Enthaltungen (einfache Mehrheit).
  \end{enumerate}
  'S Ein Antrag auf Satzungsänderung darf nur in einer Studierendenratssitzung
  abgestimmt werden, wenn er auf mindestens zwei vorherigen Sitzungen des
  Studierendenrates erörtert wurde'. Wird bei der Wahl der Vorsitzenden die
  absolute Mehrheit in zwei Wahlgängen nicht erreicht, genügt im dritten
  Wahlgang die einfache Mehrheit'. Vor einer Abwahl ist eine Stellungnahme der
  WSSK einzuholen, ob ein Abweichen der gewählten Person von einem Beschluss der
  Studierendenschaft oder ihrer Organe festgestellt werden kann; betrifft die
  Abwahl ein WSSK-Mitglied, nimmt dieses an Beratung und Beschluss der
  Stellungnahme nicht teil.

  'S Personalangelegenheiten müssen geheim, alles andere soll namentlich abgestimmt
  werden.

  'S Der Studierendenrat wird spätestens drei Wochen nach Beginn seiner Wahlperiode
  vom bisherigen Studierendenratspräsidium zur konstituierenden Sitzung
  einberufen'. Ort und Zeit der konstituierenden Sitzung sind mindestens eine
  Woche vorher bekannt zumachen'. Auf der konstituierenden Sitzung sollen sich
  die Kandidat*innen für sind das Studierendenratspräsidium, die WSSK und die
  Exekutive vorzustellen'. Diese Satzung und die Geschäftsordnung des
  Studierendenrates können auf der konstituierenden Sitzung nicht geändert
  werden'. Bis zur Wahl eines neuen Studierendenratspräsidiums leitet ein
  bisheriges Mitglied des Studierendenratspräsidiums oder,  sofern diese
  verhindert sind, ein bisheriges WSSK-Mitglied die Sitzung.


  \Clause{title={Geschäftsordnung des Studierendenrates}}

  'S Der Studierendenrat gibt sich eine Geschäftsordnung, die das Nähere regelt,
  insbesondere
  \begin{enumerate}[\qquad 1.]
  \item den Sitzungsturnus,
  \item welche Gegenstände auf das Exekutivorgan übertragen werden, und
  \item Ausnahmen von der namentlichen Abstimmung.
  \end{enumerate}


  \Clause{title={Das Studierendenratspräsidium}}

  'S Das Studierendenratspräsidium vertritt den Studierendenrat gegenüber den
  anderen Organen der Studierendenschaft'. Es bereitet die
  Studierendenratssitzungen vor und nach und leitet sie'. Es ist verantwortlich
  für die Erstellung, Veröffentlichung und Archivierung der Niederschriften über
  die Studierendenratssitzungen'. Außerdem veröffentlicht es rechtzeitig die
  Verhandlungsgegenstände der nächsten Studierendenratssitzung.

  'S Das Studierendenratspräsidium besteht aus bis zu drei Personen'. Sie dürfen
  kein anderes Amt in den Organen der Studierendenschaft, außer ihrem Mandat im
  Studierendenrat, ausüben.

  'S Das Studierendenratspräsidium kann gegen Beschlüsse, Maßnahmen und Handlungen
  des AStA ein aufschiebendes Veto einlegen'. Der Gegenstand des Vetos ist auf
  der nächsten Studierendenratssitzung zu behandeln; bis zu einer Entscheidung
  des Studierendenrates über das weitere Verfahren sind die aufgeschobenen
  Beschlüsse, Maßnahmen und Handlungen unwirksam.


\end{contract}

\addsec{Abschnitt IV: Die Fachbereiche und ihre Vertretung}
\begin{contract}

  \Clause{title={Die Fachbereiche}}

  'S Die Mitglieder eines oder mehrerer Studienfächer einer Fakultät bilden einen
  Fachbereich'. Einem Fachbereich sollen mindestens 200 Studierende angehören'.
  Der Fachbereich kann sich in Fachgruppen gliedern; die Zuordnung der
  Studienfächer zu den Fachgruppen ist in der Geschäftsordnung des Fachbereiches
  aufzuführen.

  'S Der Studierendenrat ordnet die Studienfächer den Fachbereichen durch Beschluss
  zu'. Die Zuordnung der Studienfächer zu den Fachbereichen wird in einer Anlage
  zu dieser Organisationssatzung (1. Anhang) geregelt'. Die Zuordnung ist durch
  Beschluss des Studierendenrates zu ändern, wenn neue Studienfächer
  eingerichtet werden oder wenn mindestens 20 Angehörige eines Fachbereiches
  dies beantragen'. Im Falle der Änderungen der Zuordnung, ist die betreffende
  Anlage im Einvernehmen mit dem Rektorat der Albert-Ludwigs-Universität
  Freiburg ebenfalls abzuändern.

  'S Jedes Mitglied der Studierendenschaft kann nur einem Fachbereich angehören'.
  Mit der Immatrikulation gehört sie*er dem Fachbereich ihres*seines ersten
  Hauptfachs an'. Sie*er kann ihre*seine Fachbereichsangehörigkeit im Rahmen
  ihrer*seiner Studienfächer durch schriftliche Erklärung gegenüber der WSSK
  ändern.

  'S Alle Fachbereiche einer Fakultät bestimmen im Einvernehmen das beratende
  Mitglied im Fakultätsrat.

  'S Das Nähere regelt die Geschäftsordnung des Fachbereichs, die mit absoluter
  Mehrheit der Stimmen der Fachbereichssitzung beschlossen wird'. Jede Änderung
  der Geschäftsordnung des Fachbereichs ist unverzüglich der WSSK mitzuteilen.


  \Clause{title={Änderung der Fachbereiche}}

  'S Werden neue Studienfächer geschaffen, müssen sich die Gremien möglichst bald
  nach dem Senatsbeschluss über die Errichtung der neuen Studienfächer mit der
  Fachbereichszuordnung der neuen Studienfächer befassen.

  'S Unter Berücksichtigung der Fakultät, des Instituts oder des Seminars der
  zuzuordnenden Studienfächer schlagen die Vorsitzenden oder die studentischen
  Senatsmitglieder eine Änderung des 1. Anhangs vor'. Die WSSK nimmt zu dem
  Vorschlag Stellung und leitet die Stellungnahme und den Vorschlag den
  betroffenen Fachbereichsvertretungen und dem Studierendenrat zu'. Der
  Studierendenrat muss die betroffenen Fachbereichsvertretungen bezüglich der
  Zuordnung anhören'. Der Vorschlag ist angenommen, wenn der Studierendenrat mit
  satzungsändernder Mehrheit zustimmt'. Über die Satzungsänderung kann auch in
  einer Urabstimmung entschieden werden.

  'S Wird eine Änderung des 1. Anhangs von 20 Angehörigen eines Fachbereichs
  beantragt, wird das Verfahren nach Absatz 2 entsprechend durchgeführt; der
  Antrag ersetzt dabei den Vorschlag der Vorsitzenden oder der studentischen
  Senatsmitglieder'. Für den Antrag gelten § 6 Absatz 1 und Absatz 4
  entsprechend.

  'S Entscheidet sich der Studierendenrat gegen das Votum der betroffenen
  Fachbereiche, muss er eine ausführliche Begründung abgeben'. Die WSSK nimmt zur
  Begründung Stellung'. Die betroffenen Fachbereichsvertretungen haben in jedem
  Fall das Recht eine Stellungnahme zum Beschluss des Studierendenrates
  abzugeben, welche ins Protokoll aufgenommen wird.

  'S Kommt es bei der Zuordnung von neuen Studienfächern nach zwei Vorschlägen
  nicht zu einer Zuordnung zu einem neuen oder schon bestehenden Fachbereich,
  wird der Studiengang vorläufig, bis eine Zuordnung erfolgt ist, dem kleinsten
  Fachbereich der jeweiligen Fakultät zugeordnet'. Solange ein Studiengang nicht
  endgültig einem Fachbereich zugeordnet ist, muss sich der Studierendenrat in
  jeder Sitzung mit der Zuordnung befassen.

  'S Entsteht ein neuer Fachbereich oder ändert sich die Zuordnung der
  Studienfächer zu den Fachbereichen, so sollen die neuen
  Fachbereichsvertretungen bei der nächsten Wahl gewählt werden.


  \Clause{title={Die Fachbereichsvertretungen (FaVe)}}

  'S Die*der Fachbereichsvertreter*in und maximal zehn Stellvertreter*innen bilden
  die Fachbereichsvertretung, die das exekutive Organ auf Fachbereichsebene
  bildet'. Der Fachbereich beschließt über seine Angelegenheiten auf regelmäßigen
  Fachbereichssitzungen'. Auf der Fachbereichssitzung sind alle Mitglieder des
  Fachbereichs anwesenheits-, rede-, antrags- und stimmberechtigt'. Näheres
  regelt die Geschäftsordnung des Fachbereichs.

  'S Die Fachbereichsvertretung ist Ansprechpartnerin für alle Studierenden des
  Fachbereiches und ihnen bezüglich ihrer Tätigkeiten auskunftspflichtig.

  'S Die Verhandlungsgegenstände der Fachbereichssitzung, samt der
  Verhandlungsgegenstände der nächsten Studierendenratssitzung, die vom
  Studierendenratspräsidium laut § 12 Absatz 1 veröffentlicht werden, sind
  rechtzeitig von der Fachbereichsvertretung zu veröffentlichen.

  'S Die Fachbereichssitzung ist beschlussfähig, wenn 0.75\% der Mitglieder des
  Fachbereichs, einschließlich der*des Fachbereichsvertreters*in oder
  eines*einer Fachbereichsstellvertreters*in anwesend sind, mindestens aber
  der*die Fachbereichsvertreter*in oder ein*e Fachbereichsstellvertreter*in und
  vier weitere Mitglieder des Fachbereichs'. Die Fachbereichsvertretung
  beschließt mit einfacher Mehrheit, soweit die Geschäftsordnung des
  Fachbereichs nicht etwas anderes bestimmt.

  'S Sitzungstermin und -ort der ersten Fachbereichssitzung des Semesters sind
  mindestens eine Woche vor dieser Sitzung bekannt zu machen'. Auf dieser Sitzung
  sind die weiteren Sitzungstermine und -orte für die Vorlesungszeit eines
  Semesters einheitlich festzulegen; sie sind unverzüglich bekannt zu machen.

  'S Auf Antrag von 20 Mitgliedern des Fachbereichs oder auf Beschluss der
  Fachbereichssitzung hat die Fachbereichsvertretung eine außerordentliche
  Sitzung einzuberufen'. Sie ist unter Angabe der zu behandelnden Gegenstände
  mindestens eine Woche vorher bekannt zumachen'. Der Termin einer
  außerordentlichen Sitzung kann vom regelmäßigen Termin abweichen.


  \Clause{title={Die Fachgruppen}}

  'S Hat sich ein Fachbereich in Fachgruppen gegliedert, sollen Beschlüsse der
  Fachbereichsvertretung von Angehörigen der verschiedenen Fachgruppen gemeinsam
  getroffen werden.


  'S Die Fachgruppen können eigene Geschäftsordnungen erlassen und sich im Rahmen
  der nach § 13 Absatz 2 zugeordneten Studienfächer eigenständig mit
  Angelegenheiten befassen.


  \Clause{title={Die*Der Fachbereichsvertreter*in}}

  'S Jeder Fachbereich wählt in geheimen, gleichen und freien Wahlen ein*e
  Fachbereichsvertreter*in und deren Stellvertreter*innen'. Die Geschäftsordnung
  des Fachbereichs regelt die Anzahl der Stellvertreter*innen'. Für diese Wahl
  sind nur Angehörige des Fachbereichs wählbar und wahlberechtigt'. Das Nähere
  regelt die Wahl- und Urabstimmungsordnung'. Die WSSK kann eine Nachwahl
  durchführen wenn keine Fachbereichsvertretung gewählt wurde, die gewählten
  Vertreter*innen die Wahl nicht angenommen haben, die gewählten Vertreter*innen
  ihre Wählbarkeit verlieren oder ihr Amt nach Annahme der Wahl niedergelegt
  haben.

  'S Die*der Fachbereichsvertreter*in wird von der Fachbereichsvertretung in den
  Studierendenrat entsandt und vertritt dort ihren*seinen Fachbereich und dessen
  Interessen'. Vor der Abstimmung im Studierendenrat soll die
  Fachbereichsvertretung über die im Studierendenrat behandelten Gegenstände
  diskutieren und abstimmen'. Die*der Fachbereichsvertreter*in ist an das Votum
  der Fachbereichsvertretung gebunden.

  'S Die Fachbereichsstellvertreter*innen sind die gewählten Personen, auf die nach
  der*dem Fachbereichsvertreter*in die meisten Stimmen gefallen sind'. Diese sind
  die Nachrücker*innen, falls der*die Fachbereichsvertreter*in sein*ihr Amt
  verliert.

  'S Wird die*der Fachbereichsvertreter*in von einer*einem Stellvertreter*in in
  einer Studierendenratssitzung vertreten, so muss dies dem
  Studierendenratspräsidium frühzeitig mitgeteilt werden.

  'S Der*die Fachbereichsvertreter*in oder der*die Fachbereichsstellvertreter*in
  ist der Fachbereichssitzung für seine*ihre Handlungen, insbesondere sein*ihr
  Abstimmungsverhalten im Studierendenrat, Rechenschaft schuldig'. Verletzt
  er*sie diese Pflichten oder das Imperative Mandat nach Abs. 2, so kann hierzu
  die WSSK angerufen werden'. Sie stellt nach Anhörung beider Seiten fest, ob
  eine Verletzung vorliegt.

  'S Jedes Mitglied eines Fachbereiches kann auf einer Fachbereichssitzung einen
  Antrag auf Neuwahl des*der Fachbereichsvertreter*in und aller
  Fachbereichsstellvertreter*innen stellen'. Die Fachbereichsvertretung kann nur
  in ihrer Gesamtheit abgewählt werden'. Der Antrag muss unverzüglich, in jedem
  Fall vor der nächsten Fachbereichssitzung, der WSSK vorgelegt werden'. Er wird
  auf der nächsten Fachbereichssitzung abgestimmt'. Die Sitzung kann über den
  Antrag beschließen wenn mindestens der*die Fachbereichsvertreter*in oder ein*e
  Fachbereichsstellvertreter*in und acht weitere Mitglieder des Fachbereiches,
  mindestens jedoch 0,75\% aller Mitglieder des Fachbereichs anwesend sind'. Der
  Antrag ist angenommen, wenn er mit der Zwei-Drittel-Mehrheit der Stimmen der
  Anwesenden angenommen wird'. Hat die WSSK eine Pflichtverletzung nach Abs. 5
  festgestellt, genügt die einfache Mehrheit der Stimmen'. Ist der Antrag
  angenommen, so wird die WSSK mit der Durchführung einer Neuwahl beauftragt'.
  Das Nähere regelt die Wahl- und Urabstimmungsordnung.

\end{contract}

\addsec{Abschnitt V: Die Exekutive}
\begin{contract}

  \Clause{title={Der Allgemeine Studierendenausschuss (AStA)}}

  'S Der AStA diskutiert und plant die Arbeit der Studierendenvertretung'. Er führt
  die ihm von dem Studierendenrat übertragenen Aufgaben aus.

  'S Die Mitglieder des AStA sind die Vorsitzenden und die Referate, für die ein*e
  Referent*in gewählt ist'. Das Studierendenratspräsidium nimmt beratend an den
  AStA-Sitzungen teil'. Die Anzahl der AStA-Mitglieder muss weniger als die
  Hälfte der Mitglieder des Studierendenrates betragen.

  'S Jedes Mitglied des AStA hat eine Stimme'. Der AStA beschließt grundsätzlich mit
  einfacher Mehrheit.

  'S Das Nähere regelt die Geschäftsordnung des AStA, insbesondere den
  Sitzungsturnus'. Die Geschäftsordnung des AStA bedarf der Zustimmung des
  Studierendenrates.


  \Clause{title={Der Vorstand}}

  'S Der Vorstand besteht aus mindestens einem Vorsitzenden und den
  Vorstandsreferent*innen, die gleichzeitig stellvertretende Vorsitzende sind'.
  Jede*r Vorsitzende ist gegenüber den bei der Studierendenschaft angestellten
  Personen Leiter*in der Dienststelle und unmittelbare*r Vorgesetzte*r'. Besteht
  der Vorstand aus mehreren Vorsitzenden, vertreten diese die Studierendenschaft
  gemeinschaftlich nach außen.

  'S Die Mitglieder sollen kein anderes Amt in den Organen der Studierendenschaft
  innehaben'. Sie dürfen kein anderes Amt in den zentralen Organen der
  Studierendenschaft innehaben.

  'S Mindestens ein Vorstandsmitglied soll dem Senat der Albert-Ludwigs-Universität
  Freiburg als gewähltes Mitglied angehören'. Die Vorsitzenden dürfen nicht
  gleichgeschlechtlich sein'. Die Anzahl der männlichen Vorstandsmitglieder darf
  von der Anzahl der weiblichen Vorstandsmitglieder nicht um mehr als eins
  abweichen.

  'S Die Zahl der Vorstandsreferate legt der Studierendenrat fest'. Er hat dabei den
  finanziellen Aufwand und die Maximalgröße des AStA nach § 18 Absatz 2 zu
  berücksichtigen'. Der Studierendenrat kann darüber hinaus Referent*innen das
  Recht einräumen, den*die Vorsitzende zu vertreten.


  \Clause{title={Die Referate}}

  'S Die Referate arbeiten zu bestimmten Aufgabengebieten selbständig und dauerhaft
  im Rahmen der Beschlüsse der Organe der Studierendenschaft'. Sie unterstützen
  die Organe der Studierendenschaft bei deren Arbeit'. Sie sollen gehört werden,
  bevor ein anderes Organ der Studierendenschaft einen Beschluss fasst, der
  ihren Aufgabenbereich betrifft'. Die Referate werden von Referent*innen
  vertreten.

  'S Über Einrichtung, Aufgabenbereich und Auflösung der Referate beschließt der
  Studierendenrat'. Außerdem wählt der Studierendenrat die Referent*innen und
  deren Stellvertreter*innen'. Er hat dabei den finanziellen Aufwand und die
  Maximalgröße des AStA nach § 18 Absatz 2 zu berücksichtigen.

  'S Abweichend von § 2 Absatz 4 Satz 5 führen die Referent*innen nach Ablauf ihrer
  Amtszeit die Geschäfte nicht fort.


  \Clause{title={Die autonomen Referate}}

  'S Autonome Referate sind Referate mit besonderen Rechten'. Sie arbeiten für die
  Förderung der Gleichstellung und den Abbau von Benachteiligungen im Sinne des
  § 1 Absatz 2'. Die Studierendenschaft hat je ein autonomes Referat zu den
  Aufgabenbereichen
  \begin{enumerate}[\qquad 1.]
  \item Studierende mit Beeinträchtigung und chronischer Krankheit,
  \item sexuelle Orientierung,
  \item Frauen/Gender/Geschlecht,
  \item ausländische Studierende und
  \item Studierende mit familiären Verpflichtungen.
  \end{enumerate}
  'S Die autonomen Referate können eigene Namen führen; dies ändert den
  Aufgabenbereich nicht.

  'S In ihrem Aufgabenbereich arbeiten die Referate selbständig'. Sie haben das
  Recht, zu Beschlüssen der Organe der Studierendenschaft, die ihren
  Aufgabenbereich berühren, ein Sondervotum abzugeben, das mit dem Beschluss zu
  veröffentlichen und zu archivieren ist'. Sie haben ein eigenes angemessenes
  Budget zur Erfüllung ihrer Aufgaben.

  'S Die autonomen Referate sollen Kandidat*innen zur Wahl des*der Referent*in und
  des*der Stellvertreter*in vorschlagen.


\end{contract}

\addsec{Abschnitt VII: Die Wahl-, Schlichtungs- und Satzungskommission (WSSK)}
\begin{contract}

  \Clause{title={Aufgaben}}

  'S Die WSSK ist verantwortlich für die Durchführung und Beaufsichtigung der
  Wahlen nach § 17 Absatz 1 der Fachbereichsvertreter*innen, nach § 8 der
  Abgeordneten und der Urabstimmung nach § 4 Absatz 3, insbesondere die
  Beschlussfassung über die eingereichten Wahlvorschläge oder Abstimmungsfragen
  sowie die Ermittlung und Feststellung des Wahl- oder Urabstimmungsergebnisses.

  'S Die WSSK prüft Anträge auf direktdemokratische Einflussnahme nach § 6, wie es
  die Wahl- und Urabstimmungsordung nach § 6 Absatz 4 vorsieht.

  'S Die WSSK kann von jedem Mitglied der Studierendenschaft mit der Behauptung
  angerufen werden, dass die Organe der Studierendenschaft oder von ihnen
  Gewählte in einem konkreten Einzelfall ihre Kompetenzen überschritten haben
  oder ihre Aufgaben nicht satzungsgemäß wahrgenommen haben.

  'S Die WSSK hat Stellungnahmen in den nach dieser Satzung vorgesehenen Fällen
  sowie auf Antrag eines gewählten Mitglieds eines Organs der Studierendenschaft
  über die Auslegung dieser Satzung und der in ihrem Rahmen beschlossenen
  Satzungen und Geschäftsordnungen abzugeben'. Die anderen Organe der
  Studierendenschaft sollen die Stellungnahmen über die Auslegung in ihre
  Beschlüsse miteinbeziehen.

  'S Die WSSK nimmt nach § 14 Absatz 4 Stellung zur ausführlichen Begründung des
  Studierendenrates.

  'S Die Mitglieder der WSSK sind verpflichtet, ihre Aufgaben unparteiisch und
  unvoreingenommen zu erfüllen'. Sie kann zur Erfüllung ihrer Aufgaben
  Sachverständige beratend hinzuziehen.


  \Clause{title={Zusammensetzung}}

  'S Die WSSK besteht aus fünf Mitgliedern, die mehrheitlich der Studierendenschaft
  angehören müssen'. Die Mitglieder der WSSK dürfen keinem anderen Organ der
  Studierendenschaft als gewähltes Mitglied angehören'. Von den Mitglieder der
  WSSK sollen mindestens zwei Frauen sein.

  'S Eine Wiederwahl der Mitglieder ist ein Mal möglich'. Endet die Amtszeit
  vorzeitig, kann der*die Nachfolger*in zwei Mal wiedergewählt werden.


  \Clause{title={Beschlussfassung}}

  'S Die WSSK beschließt mit absoluter Mehrheit'. Jedes Mitglied der WSSK hat das
  Recht, ein Sondervotum zu jedem Beschluss der WSSK abzugeben'. Das Sondervotum
  ist zusammen mit dem Beschluss zu veröffentlichen und zu archivieren.

  'S Eine Stellungnahme zu der Frage, ob ein autonomes Referat seinen
  Aufgabenbereich überschritten hat, kann nur im Konsens beschlossen werden'.
  Enthaltungen werden dabei nicht berücksichtigt.

  'S Das Nähere regelt die Geschäftsordnung der WSSK, insbesondere
  \begin{enumerate}[\qquad 1.]
  \item wann das Schlichtungsverfahren nach § 22 Absatz 3 beendet ist,
  \item die Fristen, innerhalb derer die WSSK Stellungnahmen abzugeben hat. Die
    Geschäftsordnung kann unterschiedliche Fristen zu den verschiedenen Anlässen
    vorsehen, die diese Satzung festlegt.
  \end{enumerate}

\end{contract}

\addsec{Abschnitt VIII: Finanzen, Aufsicht}
\begin{contract}

  \Clause{title={Allgemeines}}

  'S Für die Haushalts- und Wirtschaftsführung sowie die Aufsicht sind die
  Regelungen des § 65b LHG mit den folgenden Ergänzungen anzuwenden'. Die
  Vorschriften des Landes Baden-Württemberg zur Haushalts- und Wirtschaftsführung
  gehen dabei den Regelungen dieser Organisationssatzung vor.

  'S Für die Haushalts- und Wirtschaftsführung sind die für das Land
  Baden-Württemberg geltenden Vorschriften, insbesondere die §§ 105 bis 111 der
  Landeshaushaltsordnung, entsprechend anzuwenden; die Aufgabe des zuständigen
  Ministeriums und des Finanz- und Wirtschaftsministeriums im Sinne der §§ 105
  bis 111 der Landeshaushaltsordnung übernimmt das Rektorat der
  Albert-Ludwigs-Universität Freiburg'. Die Beschäftigten der Studierendenschaft
  unterliegen derselben Tarifbindung wie die Beschäftigten der Hochschule.

  'S Für Verbindlichkeiten haftet die Studierendenschaft mit ihrem Vermögen'. Die
  Hochschule und das Land haften nicht für Verbindlichkeiten der
  Studierendenschaft'. Studierende, die vorsätzlich oder grob fahrlässig die
  ihnen obliegenden Pflichten verletzen, insbesondere Gelder der
  Studierendenschaft für die Erfüllung anderer als der in § 65 Absatz 2 bis 4
  LHG genannten Aufgaben verwenden, haben der Studierendenschaft den ihr daraus
  entstehenden Schaden zu ersetzen'. Für die Verjährung von Ansprüchen der
  Studierendenschaft gelten § 59 des Landesbeamtengesetzes und § 48 des
  Beamtenstatusgesetzes entsprechend.

  'S Die Studierendenschaft darf keine Darlehen aufnehmen oder vergeben'. Sie darf
  ein Girokonto auf Guthabenbasis führen.


  \Clause{title={Haushalt}}

  'S Haushaltsjahr ist das Kalenderjahr.

  'S Für die Erfüllung ihrer Aufgaben erhebt die Studierendenschaft nach Maßgabe
  der Beitragsordnung angemessene Beiträge von den Studierenden'. In der
  Beitragsordnung sind die Beitragspflicht, die Beitragshöhe und die Fälligkeit
  der Beiträge zu regeln; die Beitragsordnung wird als Satzung erlassen'. Bei der
  Festsetzung der Beitragshöhe sind die sozialen Belange der Studierenden zu
  berücksichtigen'. Die Beiträge werden von der Hochschule unentgeltlich
  eingezogen.

  'S Der Studierendenrat beschließt mit der Mehrheit seiner Mitglieder darüber, ob
  statt eines Haushaltsplans (§ 106 LHO) ein Wirtschaftsplan (§ 110 LHO)
  geführt wird'. Die Vorsitzenden entwerfen zusammen mit dem Beauftragten für den
  Haushalt und dem*der Finanzreferent*in einen Haushalts- oder Wirtschaftsplan
  und legt ihn dem Studierendenrat zur Beschlussfassung vor'. Mit dem Beschluss
  über die Feststellung des Haushalts-/Wirtschaftsplans ist gleichzeitig   die
  Höhe der Beiträge für das neue Haushaltsjahr festzusetzen'. Der Studierendenrat
  hat den Haushalts-/Wirtschaftsplan bis spätestens zum 30. November vor Beginn
  des Haushaltsjahrs zu beschließen, für das der Haushalts-/Wirtschaftsplan
  gelten soll'. Das Studierendenratspräsidium leitet den beschlossenen Haushalts-
  /Wirtschaftsplan an das Rektorat der Albert-Ludwigs-Universität Freiburg zur
  Genehmigung weiter; die Genehmigung darf nur versagt werden, wenn der
  Haushalts-/Wirtschaftsplan rechtswidrig ist.

  'S Bei der Aufstellung und Ausführung des Haushalts-/Wirtschaftsplans sind die
  Grundsätze der Wirtschaftlichkeit, Sparsamkeit und der Nachhaltigkeit zu
  beachten'. Im Haushalts/Wirtschaftsplan sind den Organen der
  Studierendenvertretung, sowie den Fachbereichsvertretungen, den Referaten und
  den autonomen Referaten angemessene Mittel für die Erfüllung ihrer Aufgaben
  bereitzustellen.

  'S Für die Tätigkeit in der Studierendenvertretung kann der Studierendenrat eine
  angemessene Aufwandsentschädigung festsetzen.

  'S Nach Ende des Haushaltsjahres hat der AStA eine Jahresrechnung/einen
  Jahresabschluss aufzustellen'. Das Rektorat der Albert-Ludwigs-Universität
  Freiburg beschließt über die Entlastung der im jeweiligen Haushaltsjahr
  amtierenden Vorsitzenden'. Die Prüfbefugnis des Rechnungshofs nach § 111 der
  Landeshaushaltsordnung bleibt davon unberührt.

  'S Das Nähere regeln die Finanzordnung und die Beitragsordnung, insbesondere
  \begin{enumerate}[\qquad 1.]
  \item die Fälligkeit der Beiträge,
  \item Ausnahmen von der Beitragspflicht und Rückerstattungsverfahren,
  \item die Höhe der jeweiligen Aufwandsentschädigungen für die Mitglieder der
    Organe der Studierendenschaft.
  \end{enumerate}

\end{contract}

\addsec{Abschnitt IX: Schluss- und Übergangsbestimmungen}
\begin{contract}

  \Clause{title={Übergangsbestimmungen}}

  'S Für die ersten Wahlen zum Studierendenrat und der Fachbereichsvertreter*innen
  nach Artikel 3 § 1 Absatz 5 des Verfasste-Studierendenschafts-Gesetzes gilt
  die Wahlordnung der Albert-Ludwigs-Universität vom 27.09.2006 entsprechend mit
  folgenden Einschränkungen:
  \begin{enumerate}[\qquad 1.]
    \item Der Wahlfachbereich nach § 12 Absatz 3 dieser Satzung wird aus der
      Reihung der Fächer der Wahlfakultät bestimmt.
    \item Solange diese Satzung oder die Wahl- und Urabstimmungsordnung der
      Studierendenschaft keine Regelungen trifft, gilt § 33 der Wahlordnung mit
      der Maßgabe, dass alle Nachrücker*innen auch die Stellvertretung
      wahrnehmen können.
  \end{enumerate}


  \Clause{title={Schlussbestimmungen}}

  'S Soweit diese Satzung auf Studierendenzahlen Bezug nimmt, ist der Berechnung
  die neueste verfügbare Studierendenstatistik des Wintersemesters zugrunde
  zulegen.

  'S Diese Satzung tritt am Tage ihrer Bekanntmachung, jedoch spätestens am Tag vor
  den ersten Wahlen zum Studierendenrat und zu den Fachbereichsvertreter*innen
  in Kraft.

\end{contract}

\vspace{2cm}
Anlage: \\
1. Anhang - Übersicht über die Studienfächer der Universität Freiburg \\
\qquad (Zuordnung der Studienfächer zu den Fachbereichen)

\end{document}
