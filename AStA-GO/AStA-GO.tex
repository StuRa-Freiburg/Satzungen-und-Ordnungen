\documentclass[fontsize=12pt,parskip=half, ref=short]{scrartcl}
\usepackage[ngerman]{babel}
% \usepackage[T1]{fontenc}
\usepackage{lmodern}
\usepackage{textcomp}
\usepackage{enumerate}


\usepackage[clausemark=forceboth, juratotoc, juratocnumberwidth=2.5em]{scrjura}
\useshorthands{'}
\defineshorthand{'S}{\Sentence\ignorespaces}
\defineshorthand{'.}{. \Sentence\ignorespaces}

\usepackage{hyperref}

\hypersetup{
  colorlinks=false,
  hidelinks
}

\pagestyle{myheadings}

\begin{document}
\subject{Lesefassung}
\title{Geschäftsordnung des Allgemeinen Studierendenausschusses (AStA)}
\subtitle{der Studierendenschaft der Albert-Ludwigs-Universität Freiburg}
\date{}
\maketitle

\pagebreak

\tableofcontents

\vspace*{\fill}

Auf Grund des § 18 der Organisationssatzung der Studierendenschaft der
Albert-Ludwigs-Universität Freiburg vom 17. Mai 2013 (Organisationssatzung) hat
sich der Allgemeine Studierendenausschuss (AStA) am 22.11.2013 die nachstehende
Geschäftsordnung, zuletzt geändert durch die Änderungsordnung vom  10.05.2019,
gegeben. Der Studierendenrat hat ihr am  28.05.2019 zugestimmt.

Soweit nicht anders bezeichnet, handelt es sich bei den Sitzungen und den
Mitgliedern um die desAllgemeinen Studierenden Ausschusses (§ 18
Organisationssatzung); beim Vorstand um den Vorstand gemäß § 19
Organisationssatzung.\\
Der AStA nimmt seine Aufgaben gemäß § 18 der Organisationssatzung, § 10 der
Geschäftsordnung des Studierendenrats sowie wie in der Finanzordnung festgelegt,
wahr.

\pagebreak

\addsec{Abschnitt I:   Allgemeiner  Studierendenausschuss (AStA)}
\section*{Kapitel 1: Allgemeines}

\begin{contract}
  \Clause{title={Antragsform und -unterlagen}}
  \label{Par:Antform-unter}
  'S Anträge sind in Textform einzureichen (E-Mail genügt)'. Sie müssen einen
  Antragstext und eineBegründung enthalten.

  'S Finanzanträge müssen mit dem dafür vorgesehenen Finanzantragsformular
  gestellt werden'. Näheres regelt die Finanzordnung.

  'S Raumanträge eines Referats bedürfen weder eines schriftlichen Antrags noch
  einer Abstimmung, sofern die beantragten Räumlichkeiten nicht schon belegt
  sind.

  'S Regelmäßige Raumbelegungen können nur  semesterweise beantragt werden'.
  Sofern nichtanders vermerkt, gelten die Raumbelegungen sowohl in der
  Vorlesungszeit, als auch in der vorlesungsfreien Zeit'. Die Frist zur Stellung
  dieser Anträge setzt der AStA fest.

  'S Raumanträge von Gruppen, die kein Teil der Studierendenvertretung sind,
  bedürfen einer ausführlichen schriftlichen Begründung'. Diese soll beinhalten,
  wer den Raum wann und für welche Veranstaltung beantragt'. Grundsätzlich
  müssen sich diese Gruppen nicht persönlich im AStA vorstellen'. Die einzelnen
  Mitglieder des AStA haben aber die Möglichkeit, bei kontroversen Diskussionen
  mit einem begründeten Veto diese Regelung zu umgehen und die Gruppe
  einzuladen.

  'S Allen im StuRa vertretenen Gruppen ist ein fester wöchentlich Termin zur
  Mitgliederversammlung in einem ausreichend großen Raum der
  Studierendenvertretung zumindest während der Vorlesungszeit zu gewähren.

  'S Bei der dauerhaften Vergabe von Räumlichkeiten innerhalb der
  Studierendenvertretung sind Fachbereiche vorzuziehen.

  \Clause{title={ Vorläufige Tagesordnung }}
  \label{Par:Vorl-TO}
  'S Rechtzeitig vor der Sitzung lässt sich eine vorläufige Tagesordnung
  zusammen mit den weiterenUnterlagen für die einzelnen Tagesordnungspunkte
  online einsehen'. In die vorläufige Tagesordnung sind alle Anträge
  aufzunehmen, die die Anforderungen des \ref{Par:Antform-unter} erfüllen.

  \Clause{title={ Sitzungen}}
  \label{Par:Sitz}
  'S Die Sitzungen sind grundsätzlich öffentlich'. Die Sitzungsleitung und die
  Protokollführung wird nach dem Rotationsprinzip unter den Mitgliedern des AStA
  bestimmt'. Die Sitzungen des AStAs sollen in barrierefreien Räumen
  stattfinden'.

  'S Die AStA-Sitzungen sollen während der Vorlesungszeit wöchentlich
  stattfinden'. In der Wochevor Vorlesungsbeginn sowie in der Woche nach
  Vorlesungsende soll eine Sitzung stattfinden'. In der vorlesungsfreien Zeit
  soll spätestens alle zwei Wochen eine Sitzung stattfinden.

  'S Außerordentliche Sitzungen können vom Vorstand einberufen werden und
  außerdem auf Antrag eines Zehntels der Mitglieder des AStA'. Die Einberufenden
  begründen die Sitzung und legen Sitzungsort und Termin fest'. Die
  außerordentliche Sitzung muss nach der Festlegung unverzüglich auf der
  offiziellen Website der Studierendenvertretung der Universität Freiburg
  angekündigt werden.

  % TODO: Referenz auf Kap. 3 lösen.
  'S Zu Beginn der Sitzung ist das Protokoll der letzten Sitzung zu behandeln'.
  Anschließend ist die Tagesordnung festzustellen'. Die vorläufige Tagesordnung
  kann per Verfahrensantrag (Kapitel 3) ergänzt oder verändert werden'. Nach der
  Abhandlung der Formalia folgt ein Rundlauf, in dem jedes Mitglied und das
  Studierendenratspräsidium jeweils einen Bericht über die Tätigkeiten seit der
  vorherigen Sitzung gibt.

  'S \label{Par:Sitz.Wortmel} Wortmeldungen sollen durch das Heben einer Hand angezeigt werden'. Wer sich
  zum ersten Mal zum aktuellen Tagesordnungspunkt meldet, soll vor jenen
  aufgerufen werden, die sich schon geäußert haben; Redner*innen weiblichen und
  männlichen Geschlechts sollen abwechselnd sprechen (weich quotierte
  Erstredner*innenliste)'. Die Sitzungsleitung erteilt das Wort'. Bei direkt
  gestellten Fragen kann sie der*dem Befragten vorrangig das Wort erteilen.

  'S Die Öffentlichkeit kann mit absoluter Mehrheit der Mitglieder
  ausgeschlossen werden'.
  \label{Par:Sitz.Sperr}Mit dem Ausschluss der Öffentlichkeit kann ein
  Beschluss über die Nichtveröffentlichung der Niederschriftoder von Teilen der
  Niederschrift verbunden werden; dieser Beschluss soll befristet werden
  (Sperrfrist)'. Mitglieder des AStA, das Studierendenratspräsidium und
  Mitglieder der WSSK können nicht ausgeschlossen werden.

  \Clause{title={Niederschrift}}
  \label{Par:Nieders}
  'S Die Niederschrift soll den Verlauf der Sitzung wiedergeben, insbesondere
  die Argumente fürund wider die einzelnen behandelten Gegenstände'. Die Nennung
  von Klarnamen soll vermieden werden'. Die Niederschrift muss die Ergebnisse
  der Abstimmungen wiedergeben.

  'S Die Niederschrift ist baldmöglichst, jedoch spätestens einen Tag vor der
  nächsten Sitzung zuversenden und als vorläufig gekennzeichnet auf der
  offiziellen Website der Studierendenvertretung der Universität Freiburg zu
  veröffentlichen.

  'S In der auf den Versand folgenden Sitzung kann die Niederschrift per
  Verfahrensantrag (Kapitel3) geändert werden'. Wenn es keine Änderungsanträge
  mehr gibt, gilt die überarbeitete Fassung der Niederschrift als beschlossen'.
  Sie ist sodann zu veröffentlichen.

  'S Die Niederschriften sind mindestens fünf Jahre zu archivieren'. Die
  Sitzungsunterlagen sollen mit der Niederschrift zusammen archiviert werden.

  'S Ein Antrag auf Einsicht in die Niederschrift ist zu versagen, wenn die
  Sperrfrist nach \ref{Par:Sitz.Sperr} noch nicht abgelaufen ist und die*der
  Beantragende von der Sitzung ausgeschlossen war; dies gilt nicht für
  amtierende Mitglieder des AStA, des Studierendenratspräsidiums und der WSSK.

  \Clause{title={Nutzung elektronischer Medien}}
  \label{Par:Elek-Med}
  'S Jedes Mitglied ist selbst dafür verantwortlich, ein E-Mail-Konto für den
  Mailverkehr, der durch das Amt anfällt, vorzuhalten und den Maileingang zu
  überprüfen.
\end{contract}

\section*{Kapitel 2: Abstimmungsverfahren}
\begin{contract}

  \Clause{title={Wahlverfahren}}
  \label{Par:Wahlv}
  'S Über Bewerber*innen wird in einzelnen Wahlgängen abgestimmt'. Gewählt ist,
  wer die meisten Stimmen auf sich vereinigen kann und die erforderliche
  Mehrheit (i.d.R. einfache Mehrheit) erreicht.

  'S Über Bewerber*innen wird auf Vorschlag einer durch den AStA zu bildenden
  Bewerbungskommission entschieden'. Der Kommission gehören der*die
  Haushaltsbeauftragte*, zwei Mitglieder des Vorstands, ein*e Referent*in aus
  den autonomen Referaten, sowie eine, soweit vorhanden, weitere beschäftigte
  Person, die mit den Aufgaben der jeweiligen Position vertraut ist, an.

  'S Die Bewerbungskommission sichtet die Unterlagen und führt
  Bewerbungsgespräche durch. In der folgenden AStA-Sitzung unterbreitet die
  Kommission dem AStA einen Vorschlag hinsichtlichder Wahl der Bewerber*innen'.
  Der AStA entscheidet über den Vorschlag in geheimer Abstimmung.

  'S Bleiben Positionen bzw. Stellen frei, weil Bewerbungen abgelehnt wurden,
  können diese erst nach erneuter Ausschreibung besetzt werden.

  \Clause{title={Übrige ordentliche Abstimmungsverfahren}}
  \label{Par:Abstimmver}
  'S Änderungsanträge können während des betreffenden Tagesordnungspunkts von
  der*dem Steller*in des Hauptantrags übernommen werden; sie werden damit ohne
  Abstimmung Teil des Hauptantrags'. Änderungsanträge sind angenommen, wenn sie
  die einfache Mehrheit erreichen, auch wenn der Beschluss des Hauptantrags eine
  qualifizierte Mehrheit erfordert.

  'S Finanzanträge werden mit einfacher Mehrheit abgestimmt.

  'S Über Personenangelegenheiten ist geheim abzustimmen.

  'S Die Mitglieder stimmen in der Regel geheim ab'. Dazu wird während der
  Sitzung mit Stimmzetteln abgestimmt'. Die Stimmzettel müssen den geheim
  abgestimmten Tagesordnungspunkt erkennen lassen.
\end{contract}

\section*{Kapitel 3: Verfahrensanträge („GO-Anträge“)}
\begin{contract}

  \Clause{title={Verfahren}}
  \label{Par:Verf}
  'S Verfahrensanträge sollen wenn möglich durch das Heben beider Hände
  angezeigt werden'. Der*dem Antragsteller*in ist unmittelbar nach dem Ende des
  aktuellen Redebeitrags das Wort zuerteilen'. Verfahrensanträge sind in der
  Reihenfolge abzuarbeiten, in der sie aufgerufen werden'. Die Redeliste nach
  \ref{Par:Sitz.Wortmel} bleibt in jedem Falle unberücksichtigt, auch wenn
  mehrere Verfahrensanträge gleichzeitig gestellt werden'. Die Sitzungsleitung
  kann jederzeit einen Verfahrensantrag stellen, ohne die Hände zu heben.

  'S Verfahrensanträge sind angenommen, wenn es keinen Widerspruch gegen sie
  gibt'. Gibt es Widerspruch, kann dieser begründet werden'. Die Sitzungsleitung
  darf maximal eine Wortmeldungzur Begründung zulassen'. Dabei sind begründete
  Widersprüche formalen vorzuziehen'. Danachwird über den Antrag abgestimmt'.
  Der Antrag ist angenommen, wenn er die einfache Mehrheit der Abstimmenden
  erreicht.

  'S Verfahrensanträge sind insbesondere:
  \begin{enumerate}
    \item Antrag auf Änderung der Tagesordnung. Dieser Antrag bedarf im
      Falle einer Abstimmungder absoluten Mehrheit.
    \item Antrag auf Vertagung eines Tagesordnungspunkts: Verschiebung in
      eine andere Sitzung. Vertagungen müssen begründet werden.
    \item Antrag auf Nichtbefassung mit einem Antrag oder Tagesordnungspunkt:
      Dieser Antrag bedarf der absoluten Mehrheit der Mitglieder des AStA.
    \item Antrag auf Beschränkung der Redezeit pro Wortmeldung.
    \item Antrag auf Schließung der Redeliste: Ende der Debatte nach
      Abarbeitung der Redeliste zum aktuellen Antrag oder Änderungsantrag.
      Wortmeldungen, die  unmittelbar nach Annahme des Antrags auf Schließung
      der Redeliste angezeigt werden, sind noch in die Redeliste aufzunehmen.
    \item Antrag auf Ende der Debatte: Sofortiges Ende der Diskussion zum
      aktuellen Antrag oder Änderungsantrag ohne Abarbeitung der Redeliste.
    \item Antrag auf erneute Diskussion und Abstimmung. Dieser Antrag kann
      nur einmal pro Tagesordnungspunkt gestellt werden. Dieser Antrag kann
      nicht abgelehnt werden.
    \item Antrag auf ein außerordentliches Abstimmungsverfahren.
    \item Antrag auf geheime Abstimmung: Wird dieser Antrag von Mitgliedern
      des AStA gestellt, kann er nur mit einer 2/3-Mehrheit der Anwesenden
      abgelehnt werden.
  \end{enumerate}
\end{contract}

\section*{Kapitel 4: Kompetenzübertragungen}
\begin{contract}
  \Clause{title={Kompetenzübertragungen}}
  \label{Par:Kompüber}
  'S Der AStA kann einzelne Personen oder Personengruppen mit konkreten Aufgaben
  betrauen'. Die Beauftragten nehmen im Rahmen der Beschlusslage die Kompetenzen
  des AStA wahr'. Über Entscheidungen ist der AStA unverzüglich zu informieren'.
  Das Studierendenratspräsidium kann innerhalb von zwei Tagen nach dem Bericht
  Einspruch einlegen'. Damit gilt die Entscheidung als abgelehnt und kann in
  einer Sitzung neu verhandelt werden'. Die Regelungen von Absatz 2 bis 5
  bleiben von der Einspruchsregelung ausgenommen.

  'S Für den Fall, dass zwischen dem Zeitpunkt der Einreichung und dem Zeitpunkt
  des Bedarfs eines Raumantrags keine AStA-Sitzung stattfindet, so kann der
  Antrag dennoch durch den Raumausschuss genehmigt werden'. Dieser besteht aus
  den Mitgliedern des Vorstands, des Studierendenratspräsidiums und einem,
  jeweils für ein Semester durch den AStA gewählten, Mitglieds des AStA'.
  Wiederkehrende Raumanträge sind im oben genannten Falle automatisch genehmigt,
  sofern von den genannten Personen kein Widerspruch eingelegt wird'. Über so
  getroffene Entscheidungen ist in der nächsten AStA-Sitzung zu berichten.

  'S Über die Verwendung der Mittel aus dem Budget „Materialien
  Fahrradwerkstatt“ entscheiden die Betreuer*innen der Fahrradwerkstatt im
  Konsens'. Diese sind dem AStA rechenschaftspflichtig.

  'S Über die Verwendung der Mittel aus den Budgets „Materialienkauf zu Verkauf
  und Verleih bestimmt“ und „Bürobedarf“ entscheiden die
  Sekretariatsmitarbeitenden'. Diese sind dem AStA rechenschaftspflichtig.

  'S Über die Verwendung der Mittel aus dem Budget „Studierendenzeitung“
  entscheidet die*der Pressereferent*in gemeinsam mit ihren*seinen
  Stellvertreter*innen'. Kann kein Konsens hergestellt werden, entscheidet der
  AStA. Die Mittel müssen zweckgebunden zum Druck der Studierendenzeitung
  eingesetzt werden'. Es dürfen maximal 2/3 der im Jahresbudget vorgesehen
  Mittel innerhalbeines Semsters ausgegeben werden'. Die*der Pressereferent*in
  ist dem AStA rechenschaftspflichtig.
\end{contract}

\addsec{Abschnitt II: Schlussbestimmungen}
\begin{contract}
  \Clause{title={Abweichen von der Geschäftsordnung}}
  'S Von dieser Geschäftsordnung kann im Einzelfall mit absoluter Mehrheit
  abgewichen werden.

  \Clause{title={Inkrafttreten}}
  'S Diese Geschäftsordnung tritt am Tage der Zustimmung durch den
  Studierendenrat in Kraft.
\end{contract}

\end{document}