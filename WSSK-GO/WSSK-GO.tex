\documentclass[fontsize=12pt,parskip=half]{scrartcl}
\usepackage[ngerman]{babel}
% \usepackage[T1]{fontenc}
\usepackage{lmodern}
\usepackage{textcomp}
\usepackage{enumerate}


\usepackage[clausemark=forceboth, juratotoc, juratocnumberwidth=2.5em]{scrjura}
\useshorthands{'}
\defineshorthand{'S}{\Sentence\ignorespaces}
\defineshorthand{'.}{. \Sentence\ignorespaces}

\usepackage{hyperref}

\hypersetup{
  colorlinks=false,
  hidelinks
}

\pagestyle{myheadings}

\begin{document}
\subject{Lesefassung}
\title{Geschäftsordnung der Wahl-, Schlichtungs- und Satzungskommission (WSSK)}
\subtitle{der Studierendenschaft der Albert-Ludwigs-Universität Freiburg}
\date{}
\maketitle

\pagebreak

% \tableofcontents

% \vspace*{\fill}

Die WSSK gibt sich nach §24 Abs. 3 der Organisationssatzung der
Studierendenschaft folgende Geschäftsordnung:

\begin{contract}
  \Clause{title={Ladung}}
  'S Zu den Sitzungen der WSSK ist mindestens 3 Tage im Voraus schriftlich
  einzuladen'. Die Einladung erfolgt per E-Mail'. Sie muss eine vorläufige
  Tagesordnung enthalten.

  'S Die Ladung soll durch die Person erfolgen, die die Redeleitung am Ende der
  vorangegangen Sitzung innehatte'. Jedes Mitglied der WSSK kann zu einer
  Sitzung einladen.

  'S Sind nicht alle Mitglieder der WSSK anwesend, so ist eine Änderung der
  Tagesordnung nicht möglich'. Änderungen der Tagesordnung werden mit einfacher
  Mehrheit beschlossen.

  \Clause{title={Fristen}}
  'S Anträge an die WSSK werden im Regelfall innerhalb von 10 Tagen bearbeitet'.
  Das gilt insbesondere für einfache Organisationssatzungsauslegungen‚
  Fachbereichsänderungen und Fachbereichswechsel.

  'S In den in dieser GO speziell geregelten Verfahren oder in besonderen Fällen
  kann von der 10-Tage Frist abgewichen werden'. Die WSSK teile diese Abweichung
  unverzüglich der Antragstellerin mit.

  Die in insbesondere §6 der Organisationssatzung sowie in der Wahl- und
  Urabstimmungsordnung geregelten Fristen bleiben unberührt.

  \Clause{title={ Redeleitung }}
  'S Die Redeleitung rotiert zwischen den Mitgliedern nach Alphabet'. Von dieser
  Regel kann durch einfache Mehrheit abgewichen werden'. Tritt eine neu gewählte
  WSSK das erste Mal zusammen, leitet das lebensälteste Mitglied die Sitzung.

  'S Es wird eine „weich-quotierte“ Redeliste geführt'. Es wird nach Geschlecht
  quotiert.

  'S Vor jeder Beschlussfassung wird ein Rundlauf durchgeführt, in dem jedes
  Mitglied der WSSK ihren*seinen Standpunkt kurz darlegt.

  \Clause{title={Beschlussfassung}}
  'S Soweit in dieser Geschäftsordnung oder der Organisationssatzung der
  Studierendenschaft nicht anders geregelt, entscheidet die WSSK mit absoluter
  Mehrheit der Mitglieder des Gremiums.

  'S Im Besonderen entscheidet die WSSK im Konsens über:
  \begin{enumerate}
    \item Die Frage, ob ein Autonomes Referat seinen Aufgabenbereich
      überschritten hat (§24 Abs. 2 Organisationssatzung),
    \item Änderungen dieser Geschäftsordnung, (§9 Abs. 2 GO),
    \item Wann eine Schlichtung beendet ist (§5 Abs. 3 Satz 2 GO).
  \end{enumerate}

  'S Für die Beschlussfähigkeit müssen mindesten 3 Mitglieder anwesend sein.

  'S Mitglieder können Sondervoten abgeben'. Diese müssen spätestens 7 Tage nach
  Zugang des Protokolls eingereicht werden.

  \Clause{title={Mediationsverfahren}}
  'S Das Mediationsverfahren beginnt mit der schriftlichen Anrufung der WSSK
  durch ein Mitglied der Organe der Studierendenschaft'. In dieser Anrufung
  müssen der Konflikt und das erhoffte Schlichtungsziel dargestellt werden.

  'S Die WSSK hört die*den Anrufende*n und die anderen Betroffenen einzeln an'.
  Sie formuliert einen Lösungsvorschlag und diskutiert diesen zusammen mit den
  Betroffenen'. Beide Schritte werden so oft wiederholt, bis das Verfahren
  beendet ist'. Weitere Personen können beigeladen werden.

  'S Das Schlichtungsverfahren ist beendet, wenn
  \begin{enumerate}
    \item Die*Der Anrufende und die anderen Beteiligten sich auf einen
      Lösungsvorschlag geeinigt haben und dies durch Unterschrift unter
      denselben verkünden.
    \item Die WSSK im Konsens beschließt, dass die Schlichtung beendet ist.
  \end{enumerate}

  'S Die WSSK kann ein Kommentar zu der Schlichtung veröffentlichen.

  \Clause{title={ Auslegungs- und Schlichtungsverfahren }}
  'S Die WSSK kann von jedem gewählten Mitglied der Organe der
  Studierendenschaft angerufen werden'. Von den übrigen Mitgliedern der
  Studierendenschaft kann die WSSK nur in den Fällen 1. und 2. angerufen
  werden'. Dabei muss die Anrufung einen der folgenden Fälle darlegen:
  \begin{enumerate}
    \item Dass ein Organ der Studierendenschaft oder eine von ihm gewählte*r
      Vertreter*in sein Mandat überschritten hat.
    \item Dass ein Organ der Studierendenschaft oder eine von ihm gewählte*r
      Vertreter*in seine Aufgaben nicht satzungsgemäß wahrgenommen hat.
    \item Dass Uneinigkeit über die Auslegung der Organisationssatzung der
      Studierendenschaft sowie der Geschäftsordnungen und der Satzungen, die von
      ihr erlassen wurden, besteht.
  \end{enumerate}

  'S Die WSSK hört sich alle Beteiligten an'. Sie benennt eine*n
  Berichterstatter*in‚ die*der einen Entwurf zu der Auslegung erstellt'. Der
  Entwurf muss die entscheidungserheblichen Grundlagen darstellen und kann einen
  Auslegungsvorschlag enthalten'. Er muss den anderen Mitgliedern mindestens 2
  Tage vor der Sitzung bekannt gemacht werden'. Die anderen Mitglieder sollen
  ihre Änderungswünsche so schnell wie möglich dem gesamten Gremium bekannt
  machen.

  'S Die WSSK beschließt die Auslegung mit absoluter Mehrheit'. Die
  Ausformulierung der Auslegung kann per E-Mail bestätigt werden'. Mitglieder
  dürfen ein Sondervotum abgeben.

  'S Nach dem Beschluss wird dieser der*dem Antragsteller*in zugesandt und
  allgemein bekannt gemacht'. Gab es Sondervoten, so werden diese beigelegt bzw.
  veröffentlicht'. Für das Erstellen dieser Voten sind den anwesenden
  Mitgliedern nach der Abstimmung 7 Tage Zeit zu geben'. Bei Abwesenheit beginnt
  diese Frist mit Zugang des Protokolls'. Vor dieser Frist darf der Beschluss
  weder bekannt gemacht noch versandt werden'. Verzichten alle Mitglieder
  explizit auf ein Sondervotum, kann die Veröffentlichung unverzüglich erfolgen.

  \Clause{title={Öffentlichkeit}}
  'S  Die WSSK tagt grundsätzlich öffentlich und fällt Entscheidungen offen.

  'S Auf Antrag eines Mitglieds wird die Öffentlichkeit bei Abstimmungen
  ausgeschlossen.

  'S Durch einfache Mehrheit oder in den Fällen von § 4 Abs. 2 Satz 1 GO kann
  die Öffentlichkeit auch bei Debatten ausgeschlossen werden.

  \Clause{title={Protokoll}}
  'S Die*Der Protokollführer*in wird am Anfang jeder Sitzung bestimmt.

  'S Es wird ein Ergebnisprotokoll geführt'. Auf Antrag eines Mitgliedes muss
  ein Verlaufsprotokoll geführt werden'. Ein Verlaufsprotokoll enthält
  namentliche Abstimmungen.

  'S Nach Genehmigung wird das Protokoll in geeigneter Form veröffentlicht'.
  Nichtöffentliche Teile der Sitzungen (§7 Abs. 3 GO) werden nicht
  veröffentlicht.

  'S Die Genehmigung des Protokolls kann per E-Mail Umlaufverfahren stattfinden.

  \Clause{title={ Geschäftsordnungsänderung und In-Kraft-Treten}}
  'S  Diese Geschäftsordnung tritt durch Beschluss der WSSK mit sofortiger
  Wirkung in Kraft.

  'S Diese Geschäftsordnung kann geändert werden, insofern nicht mehr als ein
  Mitglied der WSSK gegen die Änderung dieser Geschäftsordnung stimmt.
\end{contract}
\end{document}
