\documentclass[fontsize=12pt,parskip=half, ref=short]{scrartcl}
\usepackage[ngerman]{babel}
% \usepackage[T1]{fontenc}
\usepackage{lmodern}
\usepackage{textcomp}
\usepackage{enumerate}


\usepackage[clausemark=forceboth, juratotoc, juratocnumberwidth=2.5em]{scrjura}
\useshorthands{'}
\defineshorthand{'S}{\Sentence\ignorespaces}
\defineshorthand{'.}{. \Sentence\ignorespaces}

\usepackage{hyperref}

\hypersetup{
  colorlinks=false,
  hidelinks
}

\pagestyle{myheadings}

\begin{document}
\subject{Lesefassung}
\title{Geschäftsordnung des Studierendenrats (StuRa)}
\subtitle{der Studierendenschaft der Albert-Ludwigs-Universität Freiburg}
\date{}
\maketitle

\pagebreak

\tableofcontents

\vspace*{\fill}

Auf Grund von § 11 der Organisationssatzung der Studierendenschaft der
Albert-Ludwigs-Universität Freiburg vom 17. Mai 2013 (Organisationssatzung) hat
sich der Studierendenrat am 13.05.2014 die nachstehende Geschäftsordnung
gegeben, zuletzt geändert am 06.06.2019.

Soweit nicht anders bezeichnet, handelt es sich bei den Sitzungen, dem Präsidium
und den Mitgliedern um die des Studierendenrates (§ 10, § 12 und § 8 Organisationssatzung); beim Vorstandum den Vorstand gemäß § 19 Organisationssatzung.

\pagebreak

\addsec{ Abschnitt I: Studierendenrat }

\section*{ Kapitel 1: Allgemeines }

\begin{contract}
  \Clause{title={Antragsform und -unterlagen}}
  \label{Par:Antform-unter}

  'S Anträge sind in Textform einzureichen (E-Mail genügt), dabei ist das
  hierfür vorgesehene Formular zu verwenden'. Sie müssen einen Antragstext und
  eine Begründung enthalten.

  'S Finanzanträge werden durch das dafür vorgesehene Formular gestellt'.
  Finanzanträgen muss zusätzlich eine Übersicht über die erwarteten Einnahmen
  und Ausgaben sowie eine Übersicht bereits angefragter Sponsor*innen und
  beigelegt werden.

  'S Alle Anträge müssen bis spätestens am Werktag vor der Sitzung um 12 Uhr an
  das Präsidium gesendet werden'. Ausgenommen sind Finanzanträge'. Diese sind
  bei der Finanzstelle spätestens drei Werktage vor der Sitzung um 10 Uhr per
  E-Mail oder schriftlich zu stellen'. Nachbesserungen müssen der Finanzstelle
  spätestens am Werktag vor der jeweiligen Sitzung um 10 Uhr vorliegen.

  'S Das Präsidium kann für Sitzungen in der vorlesungsfreien Zeit (§3 Abs. 3)
  abweichende Antragsfristen festlegen'. Diese sind dem Studierendenrat
  spätestens in der letzten regulären Sitzung der Vorlesungszeit bekanntzugeben.

  \Clause{title={ Versand von Sitzungsunterlagen }}
  \label{Par:Vers-Sitzunter}
  'S Eine vorläufige Tagesordnung wird zusammen mit den weiteren Unterlagen für
  die einzelnen Tagesordnungspunkte rechtzeitig vom Präsidium versandt'. In die
  vorläufige Tagesordnung sind alle Anträge aufzunehmen, die die Anforderungen
  des \ref{Par:Antform-unter} erfüllen'. Das Präsidium kann Anträge, die
  dieAnforderungen des \ref{Par:Antform-unter} nicht erfüllen, mit der
  Beschlusslage des Studierendenrates identisch sind oder   offensichtlich
  gegen   die   Bestimmungen   der Organisationssatzung   oder   dieser
  Geschäftsordnung verstoßen, zurückweisen.
  
  'S Der Versand der Unterlagen kann durch Nennung eines Links ersetzt werden,
  durch den die Unterlagen abrufbar sind.

  \Clause{title={Sitzungen}}
  \label{Par:Sitz}
  'S Die Sitzungen sind grundsätzlich öffentlich. Sie sollen in offener und
  partizipativer Atmosphäreablaufen'. Dabei ist insbesondere darauf zu achten,
  dass sich jede*r gerne beteiligt, die Geschäftsordnung eingehalten wird,
  Beleidigungen und Diskriminierungen nicht geduldet werden'. Die Sitzungen des
  Studierendenrats sollen in barrierefreien Räumen stattfinden. Mindestens ein
  Vorstandsmitglied soll bei den Sitzungen vertreten sein und über die
  Tätigkeiten der AStA-Mitglieder berichten.

  'S Der Studierendenrat legt einen Sitzungsturnus für die Semesterzeit und die
  vorlesungsfreie Zeit fest'. Das Präsidium kann bei Bedarf außerordentliche
  Sitzungen ansetzen; der Bedarf ist bei Versand der vorläufigen Tagesordnung zu
  begründen'. Außerordentliche Sitzungen sind außerdem auf Antrag eines Fünftels
  der Mitglieder einzuberufen.

  'S Während der vorlesungsfreien Zeit tagt der Studierendenrat in der ersten
  und letzten Woche derselben sowie alle vier Wochen zwischen diesen Terminen.

  'S Die vorläufige Tagesordnung kann per Verfahrensantrag (Kapitel 3) ergänzt
  werden.

  'S Wortmeldungen werden durch das Heben einer Hand angezeigt'. Wer sich zum
  ersten Mal zum aktuellen Tagesordnungspunkt meldet, soll vor jenen aufgerufen
  werden, die sich schon geäußert haben; Redner*innen weiblichen und männlichen
  Geschlechts sollen abwechselnd sprechen (quotierte Erstredner*innenliste)'.
  Letzteres genießt Vorrang'. Die Sitzungsleitung erteilt das Wort'. Bei direkt
  gestellten Fragen kann sie der*dem Befragten vorrangig das Wort erteilen.

  'S Die Öffentlichkeit kann mit absoluter Mehrheit der Stimmen der Mitglieder
  ausgeschlossen werden'. Mit dem Ausschluss der Öffentlichkeit kann ein
  Beschluss über die Nichtveröffentlichung der Niederschrift verbunden werden;
  dieser Beschluss soll befristet werden (Sperrfrist)'. Mitglieder des
  Studierendenrats, des Präsidiums, des AStA und der WSSK können nicht
  ausgeschlossen werden.

  'S Nach der Sitzung hat das Präsidium eine Abstimmungsübersicht sowie die
  Niederschrift an die Mitglieder zu versenden.

  'S Die Mitglieder übermitteln ihre Abstimmungsergebnisse durch Einreichen
  einer ausgefüllten Abstimmungsübersicht. Findet an diesem Tag eine Sitzung
  statt, müssen die Abstimmungsergebnisse bis zum Ende der jeweiligen Sitzung
  eingegangen sein'. DieAbstimmungsergebnisse können auch von den
  Stellvertreter*innen übermittelt werden'. Alternativkönnen
  Abstimmungsübersichten, die über ein E-Mail-Konto nach §5 eingehen, statt mit
  einer Unterschrift mit einem gültigen Protokoll der Fachbereichssitzung, das
  die jeweils getroffenen Entscheidungen bestätigt, eingereicht werden'. Bei
  Initiativen muss dieser Nachweis aufgrund ihres freien Mandats nicht erfolgen.

  'S Nicht berücksichtigt werden Übermittlungen, die nach dieser Frist eingehen,
  von einem Mitglied eingehen, dessen Mitgliedschaft ruht, oder nicht vom
  E-Mail-Konto nach §5 eingehen oder die Unterschrift des Mitglieds tragen.

  'S Die Frist kann per Verfahrensantrag (Kapitel 3) verlängert werden,
  insbesondere wenn vorlesungsfreie Tage die Rücksprache gemäß §17 Abs. 2
  Organisationssatzung gefährden.

  \Clause{title={Niederschrift}}
  \label{Par:Nieders}
  'S Die Niederschrift soll den Verlauf der Sitzung wiedergeben, insbesondere
  die Argumente für undwider die einzelnen behandelten Gegenstände'. Die Nennung
  von Klarnamen außerhalb der Anwesenheitsliste soll vermieden werden'. Die
  Niederschrift muss die Ergebnisse der Abstimmungen wiedergeben; bei
  namentlichen Abstimmungen ist aufzuführen, wer wie abgestimmt hat.

  'S In der auf den Versand folgenden Sitzung kann die Niederschrift per
  Verfahrensantrag (Kapitel3) geändert werden'. Wenn es keine Änderungsanträge
  mehr gibt, gilt die überarbeitete Fassung der Niederschrift als beschlossen.

  'S Die Niederschriften sind mindestens fünf Jahre zu archivieren'. Die
  Sitzungsunterlagen sollen mit der Niederschrift zusammen archiviert werden.

  'S Ein Antrag auf Einsicht in die Niederschrift ist zu versagen, wenn die
  Sperrfrist nach §3 Abs. 6S. 2 noch nicht abgelaufen ist und der*die
  Beantragende von der Sitzung ausgeschlossen war; diesgilt nicht für amtierende
  Mitglieder des Studierendenrats, des Präsidiums, des AStA und der WSSK.

  \Clause{title={Nutzung elektronischer Medien}}
  \label{Par:ElekMed}
  'S Der Versand von Unterlagen findet in der Regel per E-Mail statt'. Jedes
  Mitglied ist selbst dafürverantwortlich, ein E-Mail-Konto für den Mailverkehr,
  der durch das Amt anfällt, vorzuhalten und den Maileingang zu überprüfen.

\end{contract}

\section*{Kapitel 2: Abstimmungsverfahren}

\begin{contract}
  \Clause{title={Wahlverfahren}}
  \label{Par:Wahlver}
  'S Über die Bewerber*innen wird mit dem Schulze-Verfahren entschieden.
  (Erläuterung siehe Anhang) 

  \Clause{title={Übrige ordentliche Abstimmungsverfahren}}
  \label{Par:OrdAbstimmv}
  'S Mit Ausnahme  von Finanzanträgen  werden alle  Anträge  mit  dem
  Schulze-Verfahren abgestimmt'. Anträge, die eine einfache Mehrheit benötigen,
  sind angenommen, wenn sie den Vergleich mit „Nein“ gewinnen.

  'S Änderungsanträge können während des betreffenden Tagesordnungspunkts von
  der*demSteller*in des Hauptantrags übernommen werden; sie werden damit ohne
  Abstimmung Teil des Hauptantrags'. Über Änderungsanträge wird mit dem
  Schulze-Verfahren entschieden, dabei gelten alle Änderungsanträge die den
  direkten Vergleich gegen den Hauptantrag gewinnen als angenommen und alle
  anderen als abgelehnt.

  'S Bei Abstimmungen mit absoluter Mehrheit werden nicht abgegebene Stimmen als
  Nein-Stimmen gewertet'. Alle Stimmen, die nicht ruhen, werden einbezogen'. Ist
  ein Quorum vorgesehen, gilt ein Antrag als angenommen, wenn er den Vergleich
  mit „Nein“ gewinnt und zusätzlich mindestens dem Quorum entsprechend viele
  Stimmen vor „Nein" erhält.

  'S Finanzanträge werden mit dem Median-Verfahren abgestimmt'. Jedes Mitglied
  nennt einen Betrag, den Sie den Antragsteller*innen zur Verfügung stellen
  möchte'. Die Beträge werden der Größe nach geordnet. Ein Betrag gilt als
  unterstützt, wenn das Mitglied mindestens genauso viel zur Verfügung stellen
  möchte'. Der größte Betrag, den mindestens die Hälfte der Stimmen der
  Mitglieder unterstützen gilt als angenommen.

  'S Die Liste der ideell unterstützten Gruppen wird dem Studierendenrat einmal
  jährlich im Wintersemester vorgelegt'. Die Mitglieder haben dann die
  Möglichkeit, Gruppen erneut einzuladen, wenn Bedarf zur erneuten Klärung von
  Fragen besteht, beziehungsweise Zweifel an der ideellen Unterstützung
  aufkommen'. Sprechen sich Mitglieder mit mindestens 12 Stimmen für die erneute
  Einladung aus, so ist diese durchzuführen'. Alle Gruppen, bei denen kein
  Klärungsbedarf besteht gelten weiterhin als ideell unterstützt'. §7 Abs. 5
  bleibt unberührt.

  'S Als bei schriftlichen Abstimmungen anwesend gilt, wer rechtzeitig
  seinen*ihren Stimmzettel abgegeben hat'. Werden zu noch nicht behandelten
  Themen Stimmen abgegeben, gelten auch diese als abgegebene Stimmen'. Wird eine
  Abstimmungsfrage im Verlauf der Sitzung abgeändert, sind zur ursprünglichen
  Fassung abgegebene Stimmen ungültig'. Wird eine Abstimmungsfrage im Verlauf
  einer Sitzung abgeändert, so gelten die bereits abgegebenen Stimmen, sollten
  sie nicht geändert werden können, als Ablehnung (Reihung an letzte Stelle).

  'S Wenn der Studierendenrat in einer Legislaturperiode zu einem Gegenstand
  eine Entscheidung trifft, dann muss, bevor in dieser Legislaturperiode erneut
  eine Entscheidung zu diesem Gegenstandgetroffen werden kann, die Beschlusslage
  erneut per GO-Antrag eröffnet werden.

  \Clause{title={Außerordentliche Abstimmungsverfahren}}
  \label{Par:AusAbstimmv}
  'S Ist geheime Abstimmung beantragt oder vorgeschrieben, wird während der
  Sitzung mit Stimmzetteln abgestimmt. Geheime Abstimmungen außerhalb einer
  Sitzung sind nicht möglich'. Ist geheime Abstimmung vorgeschrieben, stimmen
  alle Mitglieder geheim ab; wird geheimeAbstimmung beantragt, stimmen die
  Abgeordneten geheim ab'. Die Stimmzettel müssen den geheim abgestimmten
  Tagesordnungspunkt erkennen lassen; bei vorgeschriebenen geheimen Abstimmungen
  zusätzlich das Stimmgewicht des abstimmenden Mitglieds'. Die Stimmzettel
  zählenzu den Sitzungsunterlagen.

  'S Die Frist, innerhalb derer die Stimme abzugeben ist, kann verkürzt werden
  (Eilantrag)'. Sprechen sich Mitglieder mit 12 Stimmen gegen den Eilantrag aus,
  darf dieser nicht angewandt werden'. Der Widerspruch ist während der Sitzung
  vorzutragen'. Der Eilantrag ist in die Abstimmungsübersicht zu übernehmen'.
  Eilanträge sind nur dann zulässig, wenn der entsprechende Antrag zuvor in der
  vorläufigen Tagesordnung nach § 2 veröffentlicht wurde.

  'S Außerordentliche Abstimmungsverfahren werden durch Verfahrensantrag
  (Kapitel 3) beantragt'. Der Antrag ist zu begründen'. Sprechen sich Mitglieder
  mit 12 Stimmen gegen ein rein schriftliches Verfahren (bspw. Umlaufverfahren
  oder Abstimmung außerhalb einer Sitzung) aus, darf dieses nichtangewandt
  werden'. Der Widerspruch ist innerhalb einer Woche nach Versand der
  Niederschrift einzureichen.

  \Clause{title={Sondervotum}}
  \label{Par:Sonderv}
  'S Jedes Mitglied des Studierendenrates kann seinen Standpunkt in einem
  Sondervotum schriftlich darlegen'. Das Sondervotum wird dann in der auf die
  Abstimmung folgenden Sitzung oder vor der Abstimmung verlesen'. Das
  Sondervotum muss dem Präsidium spätestens am Folgetag der Verlesung zugeleitet
  werden'. Andere Mitglieder können sich dem Sondervotum anschließen.

  'S Das Sondervotum ist dem Präsidium bis spätestens 14 Stunden nach der
  Sitzung zuzuleiten und wird mit der Niederschrift versandt.

\end{contract}

\section*{Kapitel 3: Verfahrensanträge („GO-Anträge“)}

\begin{contract}

  \Clause{title={Verfahren}}
  \label{Par:Verf}
  'S Verfahrensanträge sollen durch das Heben beider Hände angezeigt werden'.
  Dem*Der Antragsteller*in ist unmittelbar nach dem Ende des aktuellen
  Redebeitrags das Wort zu erteilen'. Gibt es mehrere Verfahrensanträge zur
  gleichen Zeit, wird der weitestgehende Verfahrensantrag bevorzugt behandelt
  werden'. Ansonsten sind Verfahrensanträge in der Reihenfolge abzuarbeiten, in
  der sie aufgerufen werden'. Die Redeliste nach §3 Abs. 5 bleibt in jedem Falle
  unberücksichtigt, auch wenn mehrere Verfahrensanträge gleichzeitig gestellt
  werden'. Das Präsidium kann jederzeiteinen Verfahrensantrag stellen, ohne die
  Hände zu heben.

  'S Verfahrensanträge sind angenommen, wenn es keinen Widerspruch gegen sie
  gibt'. Gibt es Widerspruch, kann dieser begründet werden'. Das Präsidium darf
  maximal eine Wortmeldung zur Begründung zulassen'. Dabei sind begründete
  Widersprüche formalen vorzuziehen'. Danach wird ohne Stimmgewichtung über den
  Antrag abgestimmt'. Der Antrag ist angenommen, wenn er die einfache Mehrheit
  der Abstimmenden erreicht.

  'S Verfahrensanträge sind insbesondere, keinesfalls aber ausschließlich:
  \begin{enumerate}
  \item Antrag auf Ende der Debatte: Sofortiges Ende der Diskussion zum
    aktuellen Antrag oder Änderungsantrag ohne Abarbeitung der Redeliste. Dieser
    Antrag bedarf der absoluten Mehrheit der anwesenden Mitglieder des
    Studierendenrats.
  \item Antrag auf Schließung der Redeliste: Ende der Debatte nach Abarbeitung
    der Redeliste zum aktuellen Antrag oder Änderungsantrag. Wortmeldungen, die
    unmittelbar nach Annahme desAntrags auf Schließung der Redeliste angezeigt
    werden, sind noch in die Redeliste aufzunehmen.
  \item Antrag auf Beschränkung der Redezeit pro Wortmeldung.
  \item Antrag auf ein außerordentliches Abstimmungsverfahren. Soweit §8
    Regelungen trifft,  gehen diese denen des §10 vor.
  \item Antrag auf Verlängerung der Abstimmungsfrist (§3 Abs. 10).
  \item Antrag auf Änderung der Tagesordnung.
  \item Antrag auf Wiedereröffnung der Beschlusslage.
  \item Antrag auf Feststellung der Beschlussfähigkeit: Dieser Antrag kann nicht
    abgelehnt werden. Das Präsidium hat sofort die Beschlussfähigkeit
    festzustellen.
  \item Antrag auf Übertragung der Beschlussfassung zu einem Gegenstand auf den
    AStA.
  \item Antrag auf Übertragung der Beschlussfassung zu einem Gegenstand auf ein
    Referat. Dieser Antrag bedarf der absoluten Mehrheit der anwesenden
    Mitglieder des Studierendenrats. Beschlüsse eines Referats sind unverzüglich
    dem Präsidium des StuRa zuzuleiten. Sie sind erst wirksam, wenn das
    Präsidium bis zur nächsten Sitzung des StuRa nicht dagegen Widerspruch
    eingelegt hat.
  \item Antrag auf Vertagung eines Tagesordnungspunkts: Verschiebung in eine
    andere Sitzung. Vertagungen müssen begründet werden.
  \item Antrag auf Nichtbefassung mit einem Antrag oder Tagesordnungspunkt:
    Dieser Antrag bedarf der absoluten Mehrheit der anwesenden Mitglieder des
    Studierendenrats.
  \end{enumerate}

\end{contract}

\section*{Kapitel 4: Kompetenzübertragungen}

\begin{contract}

  \Clause{title={ Kompetenzübertragungen }}
  \label{Par:Kompüb}
  'S Der AStA kann Beschlüsse fassen zu
  \begin{enumerate}
  \item Gegenständen, die der Studierendenrat ihm im Einzelfall überträgt,
  \item Finanzanträgen, soweit sie ihm in der Finanzordnung zugeordnet sind,
  \item allen Angelegenheiten, die Raumverfügbarkeit und -zuteilung von den
    der Studierendenschaft zugewiesenen Räumlichkeiten betreffen,
  \item der Vergabe der Stellwände an Hochschulgruppen und
    Studierendeninitiativen gemäß Stellwandvergabeordnung,
  \item der Besetzung der Stellen der Studierendenschaft,
  \item der Wahl der Mitarbeiter*innen der Fahrradwerkstatt
  \item und die Wahl der Mitglieder des Wahlprüfungsausschusses, der dezentralen
    Wahlausschüsse und der erforderlichen Wahlhelfer*innen nach §4 Abs. 2 der
    Wahl- und Urabstimmungsordnung.
  \end{enumerate}

  'S Während der Vorlesungsfreien Zeit darf der AStA über Finanzanträge, auch
  wenn die Antragsteller*innen bisher nicht ideell vom StuRa unterstützt werden,
  bis zu 700€ entscheiden'. Sprechen sich innerhalb einer Woche ab Versendung
  des Protokolls Mitglieder mit 12 Stimmengegen den Beschluss aus, so muss
  dieser im StuRa behandelt werden.

  'S Die Befugnis des Studierendenrates, eigene Beschlüsse zum selben Gegenstand
  zu fällen, wird dadurch nicht eingeschränkt. Die Beschlüsse des
  Studierendenrats sind für die Exekutive verbindlich (§ 7
  Organisationssatzung).

\end{contract}

\addsec{ Abschnitt II: Schlussbestimmungen }

\begin{contract}

  \Clause{title={ Abweichen }}
  \label{Abwe}
  'S Von dieser Geschäftsordnung kann im Einzelfall mit absoluter Mehrheit der
  Mitglieder abgewichen werden.

  \Clause{title={Inkrafttreten}}
  'S Diese Geschäftsordnung tritt am Tage ihrer Abstimmung in Kraft.

\end{contract}

\end{document}